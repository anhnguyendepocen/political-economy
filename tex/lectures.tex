\hypertarget{previewing-the-debates-economics-inequality-and-the-rise-of-populism-and-illiberalism-in-advanced-and-developing-economies}{%
\section{Previewing the debates: economics, inequality, and the rise of
populism and illiberalism in advanced and developing
economies}\label{previewing-the-debates-economics-inequality-and-the-rise-of-populism-and-illiberalism-in-advanced-and-developing-economies}}

\textbf{Rodrik}

\emph{Brief history of political economy}

\begin{itemize}
\tightlist
\item
  Mercantilism: maximize gold in sovereign coffers

  \begin{itemize}
  \tightlist
  \item
    All economic activity in service of sovereign → private greed is
    bad, market highly regulated with guilds/cartels, monopolies
  \item
    Maximize exports, minimize imports → trade surplus is good
  \item
    Politics organized around absolutist/corporatist lines, sovereign
    hands out power at their discretion
  \end{itemize}
\item
  Classical liberalism: markets and private initiative leads to
  prosperity

  \begin{itemize}
  \tightlist
  \item
    Well-being measured not by gold but by satisfaction of individual
    preferences: consumption
  \item
    Need minimal state: defense, property rights, justice
  \item
    Markets should be as free as possible

    \begin{itemize}
    \tightlist
    \item
      Close correspondence with today's libertarianism
    \item
      Want markets separated from politics
    \end{itemize}
  \end{itemize}
\item
  Welfare state: markets are not self-creating, self-regulating,
  self-stabilizing, self-legitimizing

  \begin{itemize}
  \tightlist
  \item
    Reaction to Great Depression and interwar turmoil leading to New
    Deal
  \item
    Labor organization, expansion of franchise, mass media
  \item
    Embed markets in a range of institutions: regulatory,
    redistributive, monetary/fiscal, conflict management\ldots{}
  \item
    In practice, Keynes + welfare state + industrial policy = government
    acts to structure the economy
  \item
    National rather than global system of capitalism

    \begin{itemize}
    \tightlist
    \item
      Bretton Woods regime: slow down international commerce and finance
    \end{itemize}
  \end{itemize}
\item
  Socialism: capitalism is based on exploitation of labor; history is
  history of class struggle

  \begin{itemize}
  \tightlist
  \item
    End goal is public ownership of means of production
  \item
    Socialism not incompatible with markets or with democracy
  \end{itemize}
\item
  Neoliberalism:

  \begin{itemize}
  \tightlist
  \item
    Arose from postwar boom + oil crisis, stagflation, debt crises
  \item
    Keynesianism doesn't work, markets are over-regulated and states
    overextended

    \begin{itemize}
    \tightlist
    \item
      So, must stabilize, privatize, deregulate → financialization,
      removal of price controls and trade barriers
    \end{itemize}
  \item
    Hyper-globalization, financial globalization, WTO, regional trade
    agreements, Eurozone/single market
  \end{itemize}
\item
  Different models rooted in different ideas of human nature, normative
  frames, ideas of how markets work (self-stabilizing and benign
  vs.~unstable and exploitative), conceptions of politics
\item
  This class responds to increasing \emph{productive dualism} in
  developed countries -- some sectors doing much better than others --
  which is normally a feature of developing countries

  \begin{itemize}
  \tightlist
  \item
    And to premature deindustrialization in the developing world
  \end{itemize}
\item
  Populism: claim to represent ``the people'' + rejection of restrains
  on use of executive authority

  \begin{itemize}
  \tightlist
  \item
    Political: aversion to institutions of liberalism, independent
    courts, etc.
  \item
    Economic: aversion to independent regulatory agencies, independent
    central bank, eternal restraints
  \end{itemize}
\end{itemize}

\textbf{Unger}

\begin{itemize}
\tightlist
\item
  Goal is to define the conditions of socially inclusive economic
  growth: maximum theoretical and programmatic ambition combined with
  minimum technical complication
\item
  Take up 3 issues:

  \begin{enumerate}
  \def\labelenumi{\arabic{enumi}.}
  \tightlist
  \item
    Political context of argument, especially in North Atlantic
  \item
    Address organization/alternatives to market economy (the foreground
    theme of the course)
  \item
    Address character of ideas we need to pursue this (the background
    theme of the course) aimed at a redirection of the social sciences
    themselves
  \end{enumerate}
\end{itemize}

\emph{Political context}

\begin{itemize}
\tightlist
\item
  In politics, the last great moment of realignment was in the mid-20th
  century, post-WWII

  \begin{itemize}
  \tightlist
  \item
    Established the social democratic/social liberal consensus (in the
    US, this is the New Deal)
  \item
    Was a bargain between forces that wanted reshaping and that didn't

    \begin{itemize}
    \tightlist
    \item
      The state was allowed to regulate, attenuate inequality through
      taxes and transfers, manage the economy countercyclically
    \end{itemize}
  \item
    Involved a social compact between capital and labor
  \item
    This settlement has been eroded as a result of the cumulative impact
    of:

    \begin{itemize}
    \tightlist
    \item
      The change of the dominant paradigm of production from industrial
      mass production (standardized goods and services produced by
      specialized machines), hierarchical structures, semi-skilled
      labor, factories, big corporations to the knowledge economy
    \item
      The financialization of the economy: finance detached from the
      productive agenda of society

      \begin{itemize}
      \tightlist
      \item
        Sector absorbs profits, talents, human energy and threatens to
        turn finance into a bad master, not a good servant
      \end{itemize}
    \item
      A particular form of globalization which constrains attempts to
      create alternatives in national space
    \end{itemize}
  \item
    The combined impact has been to diminish growth and confine it to
    the fringes of a new vanguard of the knowledge economy, and to
    increase economic inequality and insecurity
  \item
    Left-of-center and right-of-center parties have responded by
    promoting greater flexibility in labor markets -- which is a
    euphemism for insecurity -* and offering regulatory/redistributive
    compensation through taxes and social spending

    \begin{itemize}
    \tightlist
    \item
      This response is inadequate to master the effects of inequality
      and insecurity
    \item
      Working people feel abandoned by their political representatives
    \item
      Centrist political forces have abandoned the attempt to craft a
      strategy of socially inclusive economic growth

      \begin{itemize}
      \tightlist
      \item
        In the US today, there \emph{is} no strategy of economic growth
      \item
        The residual strategy is ``cheap money,'' executed by a central
        bank: expansive monetary and fiscal policy
      \end{itemize}
    \item
      This creates a vacuum into which populism has risen
    \end{itemize}
  \item
    Populism: old (benign) American conception was to defend the
    working-class interests
  \item
    The European definition:

    \begin{itemize}
    \tightlist
    \item
      With respect to policy: solve structural problems via
      non-structural means offered in the name of the people in an
      attempt to create a new basis of division

      \begin{itemize}
      \tightlist
      \item
        On the left: monied vs.~moneyless → expansion of redistributive
        entitlements
      \item
        On the right : foreigners vs.~natives → imposition of restraints
        on migration
      \end{itemize}
    \item
      With respect to political action, pushes limits of established
      institutions → Caesarism, anti-institutional agitation

      \begin{itemize}
      \tightlist
      \item
        Premise is that we have to choose between the cold/institutional
        and the hot/extra-institutional (between Madison and Mussolini)
      \item
        Excludes idea of institutions that raise engagement in political
        life
      \end{itemize}
    \end{itemize}
  \end{itemize}
\end{itemize}

\emph{Organization and alternatives to the market economy}

\begin{itemize}
\tightlist
\item
  In the national space, there are three themes:

  \begin{itemize}
  \tightlist
  \item
    The relation of the backward to the advanced parts of production

    \begin{itemize}
    \tightlist
    \item
      New vanguard of the knowledge economy, dedicated to permanent
      innovation but confined to fringes that exclude the majority of
      firms and workers

      \begin{itemize}
      \tightlist
      \item
        This drives stagnation and inequality
      \end{itemize}
    \item
      How do we make insular vanguardism more inclusive?
    \item
      How do we change this rather than correct it after the fact?
    \end{itemize}
  \item
    The relation of labor to capital

    \begin{itemize}
    \tightlist
    \item
      Labor force consigned to precarious employment → degeneration of
      situation of labor vs.~capital that seems incompatible with a
      sustained, inclusive rise in productivity
    \item
      How do we redirect this? In the short term, address precarity; in
      the long-term, increase potential of new production to change the
      relationship between the worker and machine

      \begin{itemize}
      \tightlist
      \item
        Potential for a different kind of relation -- will technology
        result in unemployment/underemployment through automation that
        replaces labor?
      \end{itemize}
    \end{itemize}
  \item
    The relation of finance to production

    \begin{itemize}
    \tightlist
    \item
      Indifferent in good times, destructive in bad times
    \end{itemize}
  \end{itemize}
\item
  Connected to debate over globalization: imposes convergence to a
  particular kind of market economy

  \begin{itemize}
  \tightlist
  \item
    Forbids the kind of coordination that rich countries used to become
    rich
  \item
    Enhances ability of capital to punish attempts to deviate
  \end{itemize}
\end{itemize}

\emph{Background theme: where do we get our ideas?}

\begin{itemize}
\tightlist
\item
  Largely from the discipline of economics
\item
  But we need more than the ideas it provides to understand even
  economic problems
\item
  Each of the social sciences has severed the link between insight into
  the actual and the imagination of the adjacent possible
\end{itemize}

\hypertarget{the-debates-about-economic-growth-and-development}{%
\section{The debates about economic growth and
development}\label{the-debates-about-economic-growth-and-development}}

\textbf{Unger}

Today, address:

\begin{enumerate}
\def\labelenumi{\arabic{enumi}.}
\tightlist
\item
  Basis for takeoff of Europe, basis for all subsequent expansions of
  growth
\item
  Growth ``miracles'' after initial takeoff
\item
  Emerging debate about growth and development today
\end{enumerate}

\begin{itemize}
\tightlist
\item
  Until Industrial Revolution, growth was slow and fragmented

  \begin{itemize}
  \tightlist
  \item
    Agrarian empires: Chinese, Mughal

    \begin{itemize}
    \tightlist
    \item
      There was a recurrent problem that smothered growth and
      decentralized economy
    \item
      Divided into imperial center, landholders,
      tradespeople/workers/peasants
    \item
      Rulers depended on success in constraining magnates: state needs
      base of taxes and recruitment
    \item
      Characteristic policies to limit greed of magnates
    \item
      Not squarely on the side of the people → revolution
    \item
      Repeatedly, imperial center failed to constrain landholders → fell
      apart, invaded, market order reverted to barter economy
    \end{itemize}
  \item
    In Europe, pattern broken by disintegration of Roman empire

    \begin{itemize}
    \tightlist
    \item
      State formation delayed, highly fragmented with space for
      independent personalities of towns
    \item
      Nobody planned this
    \item
      Consequence was an irreversible level of social, cultural,
      economic pluralism
    \end{itemize}
  \item
    Development of natural science that tried to understand nature
    beyond perceptual experience

    \begin{itemize}
    \tightlist
    \item
      Counterintuitive understanding of nature
    \item
      Object is insight into transformations of nature to be used for
      our own benefit
    \end{itemize}
  \item
    Technological innovation is endogenous to growth process

    \begin{itemize}
    \tightlist
    \item
      Technology is a materialization of the conduit between experiments
      in the transformation of nature and experiments in cooperation
      with each other
    \item
      Based on science and depends on opening of space by
      social/economic/cultural pluralism
    \item
      When we can express formulaically, we embody in machine
    \end{itemize}
  \item
    Underlying assumption is the relationship between
    technological/economic revolution and background/society

    \begin{itemize}
    \tightlist
    \item
      Marx: development of forces of production requires revolution in
      the mode of production
    \item
      Illusions:

      \begin{itemize}
      \tightlist
      \item
        1:1 relation between level of development and way of organizing
        economy
      \item
        Deterministic: closed list of modes of production, laws
        governing their succession in history
      \end{itemize}
    \item
      In fact, there is no 1:1 relation, but there seems to be a
      direction towards greater flexibility and plasticity

      \begin{itemize}
      \tightlist
      \item
        Trust and common purpose against different backgrounds
      \end{itemize}
    \end{itemize}
  \end{itemize}
\item
  Persistent elements of growth miracles: the mobilization, the opening,
  the shield. Ex: US in first half of the 19th century

  \begin{itemize}
  \tightlist
  \item
    Mobilization: never spontaneous, needs elite vision and strategy to
    mobilize national resources to the end of developing a new
    comparative advantage

    \begin{itemize}
    \tightlist
    \item
      In the US, this was Alexander Hamilton
    \item
      War economy without a war: massive mobilization of resources,
      physical construction of country
    \item
      Use of national debt to finance mobilization
    \item
      Overarching and selective protectionism
    \end{itemize}
  \item
    Opening: in the 2 most important sectors, agriculture and finance,
    America had a ``selective democratization'' of the market order

    \begin{itemize}
    \tightlist
    \item
      Didn't just regulate and attenuate retrospectively -- innovated in
      legal and institutional architecture
    \item
      In agriculture: rejected agrarian concentration, distributed land
      on frontier, strategic coordination between government and family
      to protect family farming against unique risks through price
      supports, land grant colleges, stockpiles

      \begin{itemize}
      \tightlist
      \item
        Promoted ``cooperative competition,'' leading to economies of
        scale
      \end{itemize}
    \item
      Finance: Andrew Jackson dissolved 2nd National Bank, prohibited
      interstate banking, created decentralized system of credit
    \item
      Constraint: innovation limited by power of entrenched interests
      (the path of least resistance)
    \end{itemize}
  \item
    Shield: protect from armed and belligerent states

    \begin{itemize}
    \tightlist
    \item
      Military: alliance between armed state and national economic
      experiment
    \item
      Commercial: engage in world economy on terms conducive to own
      national experiment

      \begin{itemize}
      \tightlist
      \item
        State must enjoy power of selectivity
      \item
        Ex: ``tiger economies'' (unlike Latin American countries)
        limited foreign capital
      \end{itemize}
    \item
      Fiscal: national rebels crafting growth miracles must be able to
      resist financial interests, domestic and foreign

      \begin{itemize}
      \tightlist
      \item
        Not the vulgar Keynesianism of counter-cyclical management --
        must be subordinated to the interest of the shield
      \item
        Fiscal realism not to win financial confidence, but so country
        can rebel against financial markets
      \item
        This has contractionary effect -- need to neutralize through
        private and public investment
      \end{itemize}
    \end{itemize}
  \end{itemize}
\item
  Most advanced practice of production (not necessarily the most
  efficient) is the vanguard

  \begin{itemize}
  \tightlist
  \item
    Until recently, productive vanguard was industrial mass production

    \begin{itemize}
    \tightlist
    \item
      Context of world division of labor: capital-intensive production
      in central economies
    \end{itemize}
  \item
    Now, productive vanguard is the knowledge economy

    \begin{itemize}
    \tightlist
    \item
      Exists in every sector as the ``intellectually dense'' portion,
      but operates as an exclusive fringe
    \item
      Its insular character leads to economic stagnation and inequality
    \end{itemize}
  \item
    New dilemma: traditional shortcut to economic growth has stopped
    working

    \begin{itemize}
    \tightlist
    \item
      Long-term growth depends on ``fundamentals'': education and
      institutions
    \item
      Shortcut: take workers and resources from less productive sectors
      and move them to the more productive sector

      \begin{itemize}
      \tightlist
      \item
        This has stopped working: consider deindustrialization
      \end{itemize}
    \item
      Why? The old vanguard is no longer the vanguard
    \item
      Race to the bottom in terms of labor and taxes to the state
    \end{itemize}
  \item
    Alternative: socially inclusive new vanguard, but knowledge economy
    is quarantined even in richest economies with the most educated
    labor force
  \item
    If dilemma can be broken, can be broken only by developing inclusive
    form of the new vanguard
  \end{itemize}
\end{itemize}

\textbf{Rodrik}

\begin{itemize}
\tightlist
\item
  Empirical facts about inequality in the knowledge economy:

  \begin{itemize}
  \tightlist
  \item
    Within firms: outsourcing/offshoring through supply chains,
    platforms/gig economy
  \item
    Across firms: within every industry, productivity between leaders
    and laggards growing
  \item
    Wage distribution: winners and losers in labor market are diverging,
    even for the low-skilled (paid more if associated with productive
    center)
  \item
    Spatial: urban centers diverging from the rest
  \end{itemize}
\item
  Secular stagnation: technological advances of today have less
  potential than 40 years ago?

  \begin{itemize}
  \tightlist
  \item
    A more straightforward explanation: the most productive technologies
    are being denied to most people, leading to a slowdown
  \end{itemize}
\end{itemize}

\textbf{Unger}

\begin{itemize}
\tightlist
\item
  Knowledge economy is multisector → shouldn't identify it with platform
  companies and the associated network effects
\item
  Depends on legal/institutional arrangements -- companies pay nothing
  for data generated by customers, and this is not neutral; it is an
  artifact of the legal/institutional background
\item
  How far can productive vanguardism extend? Reason to think it depends
  on exigent conditions, unlike mass production

  \begin{itemize}
  \tightlist
  \item
    Needs certain kind of education, accumulation of social capital
    (discretion and reciprocal trust)
  \item
    Depends on legal/institutional innovation, alternative regimes of
    property and contract
  \end{itemize}
\item
  Industrial policy, by contrast, essentially consists of buying a few
  more years
\item
  Current form of the knowledge economy is probably an example of the
  ``path of least resistance''
\end{itemize}

\hypertarget{the-challenge-and-opportunity-presented-by-todays-knowledge-economy}{%
\section{The challenge and opportunity presented by today's knowledge
economy}\label{the-challenge-and-opportunity-presented-by-todays-knowledge-economy}}

\hypertarget{the-nature-and-consequences-of-the-insular-knowledge-economy}{%
\subsection{The nature and consequences of the insular knowledge
economy}\label{the-nature-and-consequences-of-the-insular-knowledge-economy}}

\textbf{Guest speaker: Yochai Benkler}

\begin{itemize}
\tightlist
\item
  Because of information asymmetries, socialized nature of individuals,
  central role of institutions, markets are pervaded by power

  \begin{itemize}
  \tightlist
  \item
    Social relations of production/culture, etc. are structured around
    power
  \item
    Here, social = set of institutions and ideologies
  \end{itemize}
\item
  Understand postwar period as a structure of social relations with
  importance placed on expertise, managerialism, Keynesianism, big
  business, unions

  \begin{itemize}
  \tightlist
  \item
    Assembly line method of production, Hollywood system of cultural
    production
  \end{itemize}
\item
  Transition from the 1970s through the Great Recession has been toward
  self-interested individuals

  \begin{itemize}
  \tightlist
  \item
    On the right: individual rationality, on the left:
    self-actualization
  \item
    State recedes from economy
  \item
    Shapes how technologies get deployed, increases competition in labor
    markets, shifting power from labor to capital
  \item
    Financialization → internationalization
  \end{itemize}
\item
  At the macro level: readjustments in ways that redistribute power (not
  just about comparative productivity)
\item
  Markets invest in institutions, technology, ideology that allows them
  to extract rents

  \begin{itemize}
  \tightlist
  \item
    Horizontal power: decrease competition for their markets by
    eliminating innovators

    \begin{itemize}
    \tightlist
    \item
      Consider the App Store or Facebook -- learn which apps are
      successful on the platform then introduce direct competitors
    \end{itemize}
  \item
    Vertical power: over workers, suppliers, consumers
  \end{itemize}
\item
  Labor management platforms make sure workers are paid only when it's
  useful for employers → this has to be backed by labor market
  institutions

  \begin{itemize}
  \tightlist
  \item
    Consider Uber's institutional framework which enables them to
    offload all risk onto drivers
  \end{itemize}
\item
  Increase in computational capacity, decrease in productivity growth,
  decrease in business dynamism, increase in concentration, increase in
  markups, oligarchic extraction in the 1\%, wage stagnation

  \begin{itemize}
  \tightlist
  \item
    Explanation: state stepping back because of neoliberalism
  \end{itemize}
\item
  Need to find a way to harness the state to contain the market --
  shrink the domain of the market

  \begin{itemize}
  \tightlist
  \item
    Karl Polyani: everything turns into a commodity
  \item
    This only happens when the state removes the ability to act outside
    the market
  \item
    Want to decommodify basic necessities: Medicare for All, public
    savings, free college, etc.
  \item
    Goal is to alleviate imperative of selling to the market
  \end{itemize}
\item
  Restructure power: increase antitrust, labor law, employment law,
  regulation of technology

  \begin{itemize}
  \tightlist
  \item
    Can we invert the surveillance system to create accountability?
  \item
    Consider bodycams: initially used to subvert police authority, now
    they regularize it; they control the video, where it's stored, etc.

    \begin{itemize}
    \tightlist
    \item
      This is a function of the institutional setting, not the
      technology itself
    \end{itemize}
  \end{itemize}
\item
  Socially embedding production -- pre-capitalism, embedded in relations

  \begin{itemize}
  \tightlist
  \item
    Idea of market autonomous of social obligations is a new one
  \item
    Sanders' idea of worker ownership: grant 2\% per year stock to
    workers
  \item
    Recreate ``what a manager needs to do'' as a battle in the
    ideological sphere
  \end{itemize}
\item
  What role can the state play to leverage fiscal power to enable
  cooperativism?

  \begin{itemize}
  \tightlist
  \item
    Quasi-utopian vision: some form of homesteading; intervention by
    state to help construct technology to produce what they need,
    embedded in social relations that use distributed materials, not
    centralized production
  \end{itemize}
\item
  Labor-displacing vs.~labor-augmenting robots? Nothing inherent in the
  technology that pushes for one or another -- technology doesn't
  develop exogenously
\item
  Can't depend on the notion that the state is infallible
\item
  Used to be more anarchistic -- skepticism about the state and the way
  it's captured by business, military-industrial complex

  \begin{itemize}
  \tightlist
  \item
    Radical distribution of computation was first time since Industrial
    Revolution that the capital requirements of production were
    distributed in the population
  \item
    Possibility of non-market production as an alternative to the
    market/state production
  \item
    Seemed possible, but no historical force that would necessarily make
    it happen → political battles over laws are central

    \begin{itemize}
    \tightlist
    \item
      Copyright, DRM systems made it easier to enforce one-sided
      contracts; see this as an ideological battle
    \end{itemize}
  \item
    Focused too little on Google, Apple

    \begin{itemize}
    \tightlist
    \item
      This changed around 2008 with the introduction of platforms, which
      introduced points of control that let them recapture power
    \item
      State, too: propaganda, surveillance
    \end{itemize}
  \end{itemize}
\item
  Realized that we need the sustaining, fiscal power of the state, but
  we can't ignore elite capture
\end{itemize}

\textbf{Unger}

\begin{itemize}
\tightlist
\item
  Market order is cast against social solidarity

  \begin{itemize}
  \tightlist
  \item
    Terminology of ``embedding'' and ``re-embedding'' is dangerous
  \end{itemize}
\item
  Social relations combine power exchange and allegiance

  \begin{itemize}
  \tightlist
  \item
    ``Sentimentalism of equal exchange''
  \item
    Marx said: good riddance, we don't want that kind of embedding
  \end{itemize}
\item
  Unsentimental version of it: sustain high investment in people and
  capabilities, withdraw from market's uncertainty a basic package of
  security-ensuring endowments

  \begin{itemize}
  \tightlist
  \item
    But consisted in a set of insider deals to the detriment of
    outsiders
  \item
    We don't want that! We don't want to re-embed market in social
    relations in this way
  \end{itemize}
\item
  Three concerns:

  \begin{enumerate}
  \def\labelenumi{\arabic{enumi}.}
  \tightlist
  \item
    Economic security and plasticity
  \item
    Inequality: we want to change the fundamental distribution of
    advantage
  \item
    Social cohesion: money transfers not adequate social cement (without
    racial/cultural homogeneity)

    \begin{itemize}
    \tightlist
    \item
      Need collective action: innovation in the economic/political
      arrangements
    \end{itemize}
  \end{enumerate}
\item
  Two themes:

  \begin{enumerate}
  \def\labelenumi{\arabic{enumi}.}
  \tightlist
  \item
    Knowledge economy: do firms prefer technology that extracts rents
    rather than increases productivity?

    \begin{itemize}
    \tightlist
    \item
      Another explanation: there is increased productivity in vanguard,
      but the problem is that the knowledge economy doesn't spread
    \end{itemize}
  \item
    Market order, containment by state: but the market is not one thing
    → which market order? The market has no one form

    \begin{itemize}
    \tightlist
    \item
      Can we change the market order by innovating in fundamental legal
      mechanisms?
    \end{itemize}
  \end{enumerate}
\end{itemize}

\hypertarget{the-alternative-futures-of-the-knowledge-economy}{%
\subsection{The alternative futures of the knowledge
economy}\label{the-alternative-futures-of-the-knowledge-economy}}

\textbf{Rodrik}

\begin{itemize}
\tightlist
\item
  Techno-optimists say the 3rd and 4th industrial revolutions are
  comparable to the 1st and 2nd, and that the benefits can be broadly
  shared with enough investment in skills
\item
  Techno-pessimists say the scale is not comparable to the 1st and 2nd,
  and that its application is limited to a few sectors, which drives
  inequality and joblessness
\item
  In development:

  \begin{itemize}
  \tightlist
  \item
    The optimists say that the rapid diffusion of technology aids
    developing countries
  \item
    The pessimists say that skill and capital intensity and the
    high-quality institutions/regulations required make it hard for
    developing countries to catch up
  \end{itemize}
\item
  Evidence?

  \begin{itemize}
  \tightlist
  \item
    Slowdown and compartmentalization of productivity growth, gap
    between leading and lagging firms, widening spatial/regional divides
  \item
    Gap between productivity and wages (especially for nonsupervisory
    workers)
  \item
    Premature deindustrialization in developing countries (because of
    globalization and skill-intensive technical progress in
    manufacturing)
  \end{itemize}
\item
  Determinants of technological innovation:

  \begin{itemize}
  \tightlist
  \item
    What questions are asked and whose problems are solved -- this is
    what social choice is about! Technology isn't exogenous
  \item
    Economics: R\&D costs are private but benefits are social, leads to
    market failure

    \begin{itemize}
    \tightlist
    \item
      Hence, innovation support: subsidies, patent protection, seed
      funding, etc.
    \end{itemize}
  \item
    Priorities shape support policies -- what counts as R\&D, which
    innovations are patentable, etc.
  \item
    Firms prefer innovations that can be monopolized, raise entry
    barriers, enhance control over workers, have costs/risks that can be
    shifted onto others
  \item
    Unlikely to invest in innovations with social externalities
  \item
    ``High-road strategies'': technology for benefit of workers/public
    as opposed to pure profit/productivity

    \begin{itemize}
    \tightlist
    \item
      Is there a ``win-win''? Is it in the interest of firms to have
      happy workers?
    \item
      No reason to think that these incentives are universal or
      sufficiently strong, especially because of coordination failures
    \end{itemize}
  \item
    The internalized norm is that technological progress is synonymous
    with saving on labor
  \end{itemize}
\end{itemize}

\textbf{Unger}

\begin{itemize}
\tightlist
\item
  Foreground theme: future of knowledge economy, which is a set of
  practices associated with potential that presently is suppressed

  \begin{itemize}
  \tightlist
  \item
    Division of labor: production at scale with
    destandardization/customization
  \item
    Momentum in production combined with decentralization of initiative
  \item
    Potential to relax constraint of diminishing marginal returns
  \item
    Revolution in moral culture of production: replace low-trust command
    and control structure with more reciprocal trust and individual
    initiative
  \item
    Make activity of producing more like activity of discovering
  \item
    Coordination of machine and anti-machine (human being)
  \end{itemize}
\item
  Background theme: how to achieve inclusive growth?

  \begin{itemize}
  \tightlist
  \item
    Need to reckon with disappointment: we had a burst of scientific and
    technological development that promised exponential growth, but
    instead we have stagnation and inequality
  \end{itemize}
\item
  Three steps to argument:

  \begin{enumerate}
  \def\labelenumi{\arabic{enumi}.}
  \tightlist
  \item
    Causation of disappointment
  \item
    Aspects of programmatic response
  \item
    Level of transformative ambitions
  \end{enumerate}
\end{itemize}

\emph{Causation of disappointment}

\begin{itemize}
\tightlist
\item
  Want to contrast 3 narratives. The first two are the most influential,
  and these RMU will argue against
\end{itemize}

\begin{enumerate}
\def\labelenumi{\arabic{enumi}.}
\tightlist
\item
  Narrative of automation: technological dynamism has run away from us

  \begin{itemize}
  \tightlist
  \item
    Internal logic of technology is that it replaces labor instead of
    augmenting it
  \item
    Thus, the best we can do is social compensation -- e.g.~a minimum
    guaranteed income
  \item
    Objection: it has never happened before that technological advances
    diminish employment -- this is a ``lump-sum'' fallacy

    \begin{itemize}
    \tightlist
    \item
      Technological evolution is indeterminate: it doesn't have a single
      form, it is \emph{shaped}
    \item
      Whether it replaces or compensates labor depends on what we do
    \end{itemize}
  \item
    The response suggested by this narrative has no relation to the
    supply side of production. It doesn't give people good jobs
  \end{itemize}
\item
  Narrative of exhaustion: contemporary technology is inherently less
  fertile

  \begin{itemize}
  \tightlist
  \item
    Have we exhausted the low-hanging fruit?
  \item
    Seems like AI, computational biology, etc. have vast potential
  \item
    Repair to Keynesianism: expansionary fiscal and monetary policy

    \begin{itemize}
    \tightlist
    \item
      But no organic, proximate connection to modern production, c.f.
      road-building in the New Deal which had a direct connection to the
      automobile industry
    \end{itemize}
  \end{itemize}
\item
  Transformation narrative: sustained growth requires breakthroughs on
  both the supply and demand sides, and there is no automatic
  correspondence between the two

  \begin{itemize}
  \tightlist
  \item
    Keynes is not actually a general theory, it is specific to Great
    Depression
  \item
    Supply side: lift up retrograde businesses and backwards parts of
    industry by giving them access to advanced technological practices
    and to markets

    \begin{itemize}
    \tightlist
    \item
      Reach individuals too -- vast parts of the parts of the labor
      force are in insecure/precarious employment, the middle class has
      been hollowed out
    \item
      Need to reach those detached from large firms: nurse
      practitioners, IT support
    \item
      Transform into ``technologically equipped artisans'' -- not just
      skills, but support
    \item
      Then go down to lower part of labor market -- janitors,
      shelf-stackers
    \item
      Then up the ladder to medicine, law, engineering
    \end{itemize}
  \item
    Prioritize things that give access to advanced practices, power to
    share/manage assets in enterprises where they work
  \item
    In the U.S., need to refinance the state -- the aggregate tax take
    is at least 10\% lower in Europe, which is incompatible with
    reconstruction

    \begin{itemize}
    \tightlist
    \item
      Only way to do this is an indirect/regressive tax on consumption,
      which will be ``gained back'' (in terms of undoing the
      regressivity) on the spending side
    \item
      Progressives don't like this -- they are against \emph{all}
      regressive taxation
    \end{itemize}
  \item
    Demand side: prefer distribution of assets to democratization of
    credit

    \begin{itemize}
    \tightlist
    \item
      Aspiration to property-owning democracy was replaced with credit
      democracy, which led to the crisis of 2008
    \item
      Can't have sustained expansion based on popularization of debt not
      backed up by sharing of assets
    \item
      Can't replace labor power with tax-and-transfer -- labor needs
      share in power, protection against insecurity, and a share of
      assets
    \end{itemize}
  \end{itemize}
\end{enumerate}

\emph{Focus on programmatic response: one aspect of supply side}

\begin{itemize}
\tightlist
\item
  Need a project of inclusive transformation
\item
  What determines if we will break from business as usual? Education and
  access to the means of production

  \begin{itemize}
  \tightlist
  \item
    Education: knowledge economy has demanding requirements, general and
    technical

    \begin{itemize}
    \tightlist
    \item
      Prioritize depth over breadth, cooperation over individualism,
      dialectical practice
    \item
      Quality can't depend on geography or circumstance, which
      necessitates federal standards
    \end{itemize}
  \item
    Access to means of production has 3 stages:

    \begin{enumerate}
    \def\labelenumi{\arabic{enumi}.}
    \tightlist
    \item
      Broaden access to larger range of agents (firms and individuals)
      -- need a decentralized procedure to discover what works
    \item
      Need a different institutional architecture

      \begin{itemize}
      \tightlist
      \item
        Not American model of arm's-length regulation
      \item
        Not northeast Asian model of industrial policy imposed top down
      \item
        Need a third form: collaboration to the end of disseminating
        advanced practices
      \end{itemize}
    \item
      Experiments with property and contract law need to be able to
      coexist in some market order
    \end{enumerate}
  \end{itemize}
\end{itemize}

\emph{Relation of structural to fragmentary change}

\begin{itemize}
\tightlist
\item
  Programmatic proposal should: (1) mark a direction and (2) select
  initial steps to move in that direction
\item
  Constituency can't be just the historical constituency of the left:
  organized labor in capital-intensive production
\item
  Instead, need to build base, must include 4 elements:

  \begin{enumerate}
  \def\labelenumi{\arabic{enumi}.}
  \tightlist
  \item
    Workers in traditional capital-intensive sectors
  \item
    Precarious workers
  \item
    Small business class
  \item
    Rank-and-file of professional business class (which is increasingly
    separate from plutocratic elite)
  \end{enumerate}
\item
  Unifying theme: become bigger together by remaking institutional
  arrangements of the economy and the state
\end{itemize}

\hypertarget{the-market-order-and-its-reconstruction}{%
\section{The market order and its
reconstruction}\label{the-market-order-and-its-reconstruction}}

\hypertarget{economics-institutions-and-the-possibility-of-alternative-institutional-forms-of-the-market-economy}{%
\subsection{Economics, institutions, and the possibility of alternative
institutional forms of the market
economy}\label{economics-institutions-and-the-possibility-of-alternative-institutional-forms-of-the-market-economy}}

\textbf{Rodrik}

\begin{itemize}
\tightlist
\item
  Do contemporary market economies depend on a particular form of
  institutions?
\item
  Institutions are the ``rules of the game in a society''

  \begin{itemize}
  \tightlist
  \item
    Formal: law, regulations
  \item
    Informal: norms, patterns, moral codes
  \item
    Not organizations in the sense that they don't need a physical form
  \end{itemize}
\item
  A functional typology of market-supporting institutions:

  \begin{itemize}
  \tightlist
  \item
    Market-creating: property rights, contract enforcement
  \item
    Market-regulating: antitrust, environment/safety regulations, labor
    market institutions, ``industrial policies'' = correction of
    market/coordination failure
  \item
    Market-stabilizing: monetary, fiscal, currency arrangements =
    macroeconomic management
  \item
    Market-legitimizing: redistribution, social insurance, political
    democracy
  \end{itemize}
\item
  Don't want to mistakenly associate these functions with forms: there's
  no unique mapping between the two because of local specificity

  \begin{itemize}
  \tightlist
  \item
    Ex: macroeconomic stability doesn't necessarily imply independent
    central banks
  \end{itemize}
\item
  We have an objective like productive efficiency (static and dynamic)

  \begin{itemize}
  \tightlist
  \item
    That implies certain universal principles: property rights,
    incentives, the rule of law
  \item
    But not particular arrangements: what type of property rights? Is
    there a role for industrial policy?
  \end{itemize}
\item
  Economics can elaborate the link between these universal principles
  and institutional arrangements
\end{itemize}

\textbf{Unger}

\begin{itemize}
\tightlist
\item
  Institutions are norms, rules, and standards that govern practice in a
  particular dimension of social life, informed by concepts of what
  those relations can and should be like in that domain
\item
  Importantly, they are not \emph{things}. This distinction means:

  \begin{itemize}
  \tightlist
  \item
    The arrangements are mediated by representations (ideas) about what
    the relations can and should be like

    \begin{itemize}
    \tightlist
    \item
      Law is the institutional form of the life of the people
    \end{itemize}
  \item
    They tilt the scale of experience in a particular direction

    \begin{itemize}
    \tightlist
    \item
      C.f. liberal political thinking, in which the aspiration is to
      form a framework of rights that is neutral with respect to
      sectarian visions of the ``good''
    \item
      But every institutional order encourages and discourages
      particular forms of relations and social life
    \item
      There is no neutral -- and this concept is dangerous because it is
      usually invoked in defense of entrenched ideas
    \item
      But neutrality is the perverted (false, dangerous) form of the
      idea that society should be open and corrigible in the light of
      experience. These are the legitimate counterparts to the idea of
      neutrality
    \end{itemize}
  \item
    They are artifacts in that we made them; we should thus try to
    understand them from the inside, not from the outside as we do
    natural phenomena

    \begin{itemize}
    \tightlist
    \item
      They are a kind of ``frozen politics'': the ordinary struggle of
      politics is temporarily interrupted and contained in institutions
    \end{itemize}
  \item
    They are not univocal: they exist ``more or less''

    \begin{itemize}
    \tightlist
    \item
      They can be organized to insulate themselves against challenge and
      change

      \begin{itemize}
      \tightlist
      \item
        Then they appear to be facts about the universe -- Marx and
        Hegel described them as the ``alienated parts of ourselves''
      \end{itemize}
    \item
      Or we can organize them so they revise themselves
    \end{itemize}
  \end{itemize}
\item
  In economics, the most important debate is about the institutional
  form of the market economy

  \begin{itemize}
  \tightlist
  \item
    In the 19th century, both Marxists and conservative jurists shared
    the belief that the market order has built-in institutional content:
    you buy the whole package

    \begin{itemize}
    \tightlist
    \item
      This is the Marxist conception that capitalism is a system with a
      predefined architecture
    \end{itemize}
  \item
    But when you try to determine the architecture of the market economy
    from first principles, there are choices to be made

    \begin{itemize}
    \tightlist
    \item
      The market has no single, natural, necessary legal form
    \item
      This was the main achievement of legal theory from approx.
      1850-1950, exemplified in the US by Oliver Wendell Holmes
    \end{itemize}
  \item
    This idea -- that the market doesn't have a natural form -- can be
    inferred at the most abstract level. There are 2 aspects of a market
    order:

    \begin{enumerate}
    \def\labelenumi{\arabic{enumi}.}
    \tightlist
    \item
      The absolute level of economic decentralization (the number of
      agents entitled to bargain)
    \item
      The absoluteness of control over resources (are property rights
      eternal, hereditary, etc.)

      \begin{itemize}
      \tightlist
      \item
        Unified property rights are an invention of the 19th century
      \item
        In most legal traditions, the normal form of property rights
        involves a disaggregation of the component parts, invested in
        various stakeholders
      \item
        This control can be fragmentary, relative, limited, and
        temporary
      \end{itemize}
    \end{enumerate}
  \item
    Conservative thinking usually postulates that decentralization and
    control go together, but the first can be advanced by limiting the
    latter
  \end{itemize}
\item
  How has economics approached the problem of the institutional form of
  the market? In three main ways:

  \begin{enumerate}
  \def\labelenumi{\arabic{enumi}.}
  \tightlist
  \item
    ``Pure'' economics: started by the marginalist theorists of the 19th
    century and achieved by the general equilibrium theorists of the
    mid-20th century

    \begin{itemize}
    \tightlist
    \item
      Free of any institutional message -- it is a form of logic, an
      analytical apparatus
    \item
      Can be applied to a command economy, which implies that the choice
      among these systems is a political one
    \end{itemize}
  \item
    ``Fundamentalist'' economics: the market order does have a legal and
    institutional form (this is precisely the idea that the 19th-century
    jurists subverted)

    \begin{itemize}
    \tightlist
    \item
      Associated with ideas of Hayek: if Robinson Crusoe traded for long
      enough, he would reproduce the entire 19th-century system of
      German private law
    \end{itemize}
  \item
    ``Equivocating'' economics: exemplified by the American followers of
    Keynes who rendered his doctrines palatable by defanging them

    \begin{itemize}
    \tightlist
    \item
      Seeks to establish law-like correlations between large-scale
      economic aggregates, e.g.~the Phillips curve
    \item
      People challenge them by saying that these correlations depend on
      institutional details (like unemployment insurance or other social
      benefits); if the institutions change, then the correlations fall
      apart

      \begin{itemize}
      \tightlist
      \item
        But they can disregard these challenges by conceding the point
        and claiming that they are just doing analysis in the status quo
      \end{itemize}
    \item
      They confuse stability and stagnation with lawfulness
    \end{itemize}
  \end{enumerate}
\item
  Placing this in the context of the history of ideas:

  \begin{itemize}
  \tightlist
  \item
    Classical European social theory (Marxism) had a structural bent: to
    distinguish the surface from the regime that shapes routines and
    conflicts
  \item
    But they surrounded this instinct with deterministic illusions that
    corrupted it; for example, within Marxism, the ideas that:

    \begin{itemize}
    \tightlist
    \item
      There is a closed list of basic regimes (modes of production):
      capitalism, feudalism, socialism, etc.
    \item
      Each of these is a complete individual system

      \begin{itemize}
      \tightlist
      \item
        This means there are only 2 types of politics: reformist
        management and revolution
      \item
        It excludes change that is structural but piecemeal
      \end{itemize}
    \item
      There are laws governing the succession of these regimes: history
      has a project, therefore we don't need one
    \end{itemize}
  \item
    Contemporary social theory and policy has abolished structural
    ideas, which results in naturalizing the established arrangements
    and presenting them as the outcome of ``best practice''
  \item
    Thus, we currently have no usable idea of structural
    change/structural alternatives
  \end{itemize}
\item
  Addressing DR's proposal of universal principles: they are a
  (mistaken) attempt to find a middle ground between pure economics,
  which is impotent, and fundamentalist/equivocating economics, which
  are powerful but corrupted

  \begin{itemize}
  \tightlist
  \item
    They try to take something off the table, with the idea that this is
    uncontroversial, that everyone can agree -- but RMU doesn't
  \item
    Moreover, the world doesn't. So is the world ill-advised? RMU says
    no
  \item
    Example: does the goal of productive efficiency imply the rule of
    law and property rights?

    \begin{itemize}
    \tightlist
    \item
      The world seems to say no
    \item
      Differential mastery of rules about property means that some
      groups manipulate them better than others, which leads to
      (sometimes violent) redistribution
    \item
      Who will have property? And what are the benefits they get?

      \begin{itemize}
      \tightlist
      \item
        Every tax on capital is a debate about this
      \item
        Should investors be rewarded, or stakeholders (workers) too?
        This is not a settled question
      \item
        Consider the law of restitution and unjust enrichment -- should
        people be rewarded if they do something to enrich others, even
        if this wasn't in a contract ex ante?
      \end{itemize}
    \end{itemize}
  \item
    Another example: does the goal of macroeconomic and financial
    stability imply ``sound money'' policies (i.e.~don't generate
    liquidity in excess of money demand?)

    \begin{itemize}
    \tightlist
    \item
      The right wing disagrees with this principle: they say the only
      effective way to achieve this objective is to have a metallic base
      for money
    \item
      This is the neoclassical synthesis: Keynesian economics downsized
      (macroeconomics) and superimposed on marginalist economic theory
      (microeconomics)
    \item
      Progressives say the principle of sound money is a principle for a
      world that no longer exists

      \begin{itemize}
      \tightlist
      \item
        Today, we're closer to the world Keynes was writing about: there
        are large pools of liquid capital being hoarded, underinvestment
      \end{itemize}
    \end{itemize}
  \item
    Final example: does the goal of distributive justice and poverty
    alleviation imply targeting and universality?

    \begin{itemize}
    \tightlist
    \item
      Aren't these principle and counter-principle? You can justify
      anything with this
    \item
      The European social democrats and FDR rejected targeting in favor
      of universal programs; LBJ began targeting with his war on poverty

      \begin{itemize}
      \tightlist
      \item
        This turned out to be harmful for the poor because in
        political/economic downturns, the benefits can be turned off or
        downsized
      \end{itemize}
    \item
      Universality is a social compact with normative content; it should
      be privileged over the principle of targeting
    \end{itemize}
  \end{itemize}
\item
  Five theses for thinking about alternative institutions:

  \begin{enumerate}
  \def\labelenumi{\arabic{enumi}.}
  \tightlist
  \item
    We need to liberate ourselves from the 19th-century idea of both
    liberals and socialists that there is a predetermined harmony
    between the institutional requirements for the development of
    productive power (i.e.~``forces of production'') and the liberation
    from hierarchy, especially class struggle

    \begin{itemize}
    \tightlist
    \item
      Today, we risk a dogmatic pessimism: that there is an insoluble
      contradiction between these things
    \item
      But there is a zone of potential overlap in which we seek to find
      institutional content specific to our context
    \end{itemize}
  \item
    The focus of ideological controversy is changing: before, it was the
    market vs.~the state, but now we need to find alternative forms of
    economic/social pluralism that go beyond the North Atlantic social
    democratic vision
  \item
    We should have a special interest in the deepening and diffusion of
    the most advanced productive practice which today is the knowledge
    economy

    \begin{itemize}
    \tightlist
    \item
      This means promoting a high-energy democracy and an education
      system that allows individuals both to resist their context and
      operate within it
    \end{itemize}
  \item
    We want institutions that are denaturalized, not ones that resist
    challenge and change

    \begin{itemize}
    \tightlist
    \item
      We want structural alternatives, not a structural dogmatism
    \item
      We want to deny our institutions the ``last word''
    \end{itemize}
  \item
    We need to be attentive to a particular dialectic: we want to make
    the individual secure in a haven of safeguards and endowments, which
    we do by taking these off the table in day-to-day politics under a
    rhetoric of ``fundamental rights''

    \begin{itemize}
    \tightlist
    \item
      But the purpose of this arrangement is to throw everything in
      society and economy open to change
    \item
      The liberal discourse has the first but not the second
    \item
      We need to figure out how to institutionally arrange for that
      ``storm'' of change
    \end{itemize}
  \end{enumerate}
\end{itemize}

\hypertarget{the-market-order-and-inequality-the-contrast-between-corrective-redistribution-and-institutional-change-influencing-the-primary-distribution-of-economic-advantage}{%
\subsection{The market order and inequality: the contrast between
corrective redistribution and institutional change influencing the
primary distribution of economic
advantage}\label{the-market-order-and-inequality-the-contrast-between-corrective-redistribution-and-institutional-change-influencing-the-primary-distribution-of-economic-advantage}}

\textbf{Rodrik}

\begin{itemize}
\tightlist
\item
  Contrast redistribution through tax-and-transfer with changing
  institutions that influence economy
\item
  Rise in inequality in Europe looks more moderate than in US

  \begin{itemize}
  \tightlist
  \item
    Why? Technology and globalization trends are the same

    \begin{itemize}
    \tightlist
    \item
      Way economy is organized, more extensive role of state\ldots{}
    \end{itemize}
  \item
    Difference shows how institutions and policies moderate
    globalization and technological shocks
  \end{itemize}
\item
  Three stages:

  \begin{enumerate}
  \def\labelenumi{\arabic{enumi}.}
  \tightlist
  \item
    Pre-production: people enter economy with endowments (networks,
    education, access to financial capital)
  \item
    Production: system of production creates market outcomes, produces
    income/wealth distribution
  \item
    Post-production
  \end{enumerate}

  \begin{itemize}
  \tightlist
  \item
    Can distinguish policies that act at these stages

    \begin{itemize}
    \tightlist
    \item
      Ex: education, UBI are pre-production
    \item
      Taxes are post-production
    \end{itemize}
  \item
    Productivist policies intervene directly in stage of production,
    employment, innovation
  \end{itemize}
\item
  Three targets:

  \begin{enumerate}
  \def\labelenumi{\arabic{enumi}.}
  \tightlist
  \item
    Help people at the bottom
  \item
    Address wealth concentration at the top
  \item
    Target middle class insecurity, precarity
  \end{enumerate}
\item
  Traditional focus: investment in health, public education
  (pre-production) and redistribution, social safety net
  (post-production)

  \begin{itemize}
  \tightlist
  \item
    Increasing discussion of policies that target production
  \end{itemize}
\item
  Focus on productivist policies addressing middle class: ``good jobs''

  \begin{itemize}
  \tightlist
  \item
    Good jobs: secure employment with benefits, give workers sense of
    satisfaction
  \item
    3 categories:

    \begin{enumerate}
    \def\labelenumi{\arabic{enumi}.}
    \tightlist
    \item
      Address productivity of existing jobs and firms
    \item
      Increase number of good jobs by supporting startups, expansion;
      attract investment from firms elsewhere

      \begin{itemize}
      \tightlist
      \item
        Currently takes place in distorted form: tax subsidies create
        ``race to the bottom''
      \end{itemize}
    \item
      Workforce development programs, continuous investment in skills
    \end{enumerate}
  \end{itemize}
\item
  Kind of industrial policy: government intervenes to affect structure
  of production and direction of technological change

  \begin{itemize}
  \tightlist
  \item
    Traditional type (East Asian model of industrialization): top-down
    incentives -- promote specific sectors (shipbuilding, steel), use
    specific instruments (export, credit subsidies)

    \begin{itemize}
    \tightlist
    \item
      Presumes ex-ante knowledge and competence on part of public
      agencies
    \end{itemize}
  \item
    More contemporary: ongoing processes, strategic collaboration across
    levels. Design principles:

    \begin{enumerate}
    \def\labelenumi{\arabic{enumi}.}
    \tightlist
    \item
      Embeddedness: knowledge embedded in activities of participants
    \item
      Institutional capacity: ability to revise policy
    \item
      Clear benchmarks of what constitutes success
    \item
      Broader accountability
    \end{enumerate}
  \end{itemize}
\item
  Four advantages of this way of thinking:

  \begin{enumerate}
  \def\labelenumi{\arabic{enumi}.}
  \tightlist
  \item
    Structural approach: focuses directly on employment

    \begin{itemize}
    \tightlist
    \item
      Transition away from welfare state model
    \end{itemize}
  \item
    Little market/state dichotomy
  \item
    No dichotomy between equity and efficiency
  \item
    Dichotomy between reform and revolution also disappears
  \end{enumerate}
\end{itemize}

\textbf{Unger}

\emph{Philosophical prelude}

\begin{itemize}
\tightlist
\item
  How should we think about inequality? Ideological debate of last two
  centuries: ``shallow equality'' vs.~``shallow freedom''

  \begin{itemize}
  \tightlist
  \item
    Left gives priority to equality, right gives priority to freedom
  \end{itemize}
\item
  ``Shallow'' because they are pursued against background of established
  institutional arrangement
\item
  Theories of distributive justice (e.g.~Rawls)

  \begin{itemize}
  \tightlist
  \item
    On their face, seem radically egalitarian
  \item
    But they are philosophical prop to compensatory redistribution
  \item
    Attempts to achieve equality via redistribution after the fact are
    limited
  \item
    Pragmatic residue: attempt to attenuate inequality generated in the
    market through retrospective tax-and-transfer
  \end{itemize}
\item
  Alternatives? Deep equality: give priority to equality of
  outcome/circumstance

  \begin{itemize}
  \tightlist
  \item
    ``We're all poor, but we're poor together''

    \begin{itemize}
    \tightlist
    \item
      Not our aim
    \end{itemize}
  \end{itemize}
\item
  Deep freedom: Bring experience of ordinary people to higher level of
  intensity, scope, capability

  \begin{itemize}
  \tightlist
  \item
    Struggle against inequality is subsidiary to this larger objective
  \item
    Narrower and deeper than equality of opportunity
  \item
    Equality of \emph{empowerment} -- opportunity to act
  \item
    Conservatives say it is natural for life to be small -- exceptions
    are heroes, thinkers, saints, entrepreneurs
  \item
    Progressives: become bigger only if we become bigger together

    \begin{itemize}
    \tightlist
    \item
      Resist inequality because it is a constraint on this shared
      empowerment
    \item
      Method: structural change in institutional arrangements,
      ideological assumptions

      \begin{itemize}
      \tightlist
      \item
        Necessarily piecemeal
      \end{itemize}
    \end{itemize}
  \end{itemize}
\end{itemize}

\emph{Historical prelude}

\begin{itemize}
\tightlist
\item
  In medium to long term, equality \emph{among} nations affected most by
  diffusion of most advanced practice of production

  \begin{itemize}
  \tightlist
  \item
    Short term, migration
  \end{itemize}
\item
  Long term, equality \emph{within} nations affected most by ability to
  share in most advanced practice of production

  \begin{itemize}
  \tightlist
  \item
    Short to medium term, war
  \item
    How can we achieve war economy without a war?

    \begin{enumerate}
    \def\labelenumi{\arabic{enumi}.}
    \tightlist
    \item
      Growth: massive mobilization of national resources
    \item
      Plasticity: large-scale mobilization together with radical
      institutional innovation

      \begin{itemize}
      \tightlist
      \item
        Create economic and political institutions that facilitate their
        own revision
      \end{itemize}
    \item
      Solidarity: elites recognize that sacrifices must be rewarded

      \begin{itemize}
      \tightlist
      \item
        Peacetime equivalent: multiplication of forms of collective
        action within and outside the economy
      \end{itemize}
    \end{enumerate}
  \end{itemize}
\end{itemize}

\emph{Relation of production to equality}

\begin{itemize}
\tightlist
\item
  Two options: diminish inequality by correcting inequalities generated
  in present market order; or diminish inequality by innovating in
  arrangements that shape fundamental distribution of advantage

  \begin{itemize}
  \tightlist
  \item
    Initiatives that affect primary distribution have more promise than
    attempt to correct after-the-fact
  \item
    If we attempt to correct after the fact, we establish tension
    between established economic incentives and egalitarian commitment:
    ``equity vs.~efficiency''
  \item
    This idea conflicts with certain orthodoxy

    \begin{itemize}
    \tightlist
    \item
      Piketty compares direct redistribution (e.g.~mandate higher return
      to labor by increasing nominal wage) to fiscal redistribution

      \begin{itemize}
      \tightlist
      \item
        Says undermines price system as system for efficient allocation
        of resources
      \item
        Direct redistribution results in less employment, preference for
        substitution of labor by capital
      \end{itemize}
    \item
      But doesn't consider extent to which fiscal redistribution can
      disorganize the real economy
    \end{itemize}
  \end{itemize}
\item
  In influencing primary distribution, three elements have priority:

  \begin{enumerate}
  \def\labelenumi{\arabic{enumi}.}
  \tightlist
  \item
    Status of labor vis-a-vis capital

    \begin{itemize}
    \tightlist
    \item
      Need different legal regime to reconcile genuine flexibility in
      labor market with security
    \item
      Future of wage labor as predominant form of free work (c.f.
      self-employment and cooperation)
    \item
      Consequence of evolution of technology for substitution of labor
    \end{itemize}
  \item
    Empowerment through productive uplift

    \begin{itemize}
    \tightlist
    \item
      Cooperative competition among producers
    \item
      Aid to middle part of job distribution
    \end{itemize}
  \item
    Education

    \begin{itemize}
    \tightlist
    \item
      Character/quality of general education: form of education that can
      empower, prefer selective depth to encyclopedic superficiality,
      approach all subjects dialectically
    \item
      Technical/vocational training: analytic/technical capabilities
      rather than job- or machine-specific skills
    \end{itemize}
  \end{enumerate}
\end{itemize}

\emph{Relation of innovation in primary distribution to after-the-fact
corrective policies}

\begin{itemize}
\tightlist
\item
  Three principles:

  \begin{enumerate}
  \def\labelenumi{\arabic{enumi}.}
  \tightlist
  \item
    Change in arrangements that shape primary distribution has priority
    over attempts to correct after-the-fact
  \item
    Common mistake: separate revenue and spending side and focus on
    progressive profile of revenue

    \begin{itemize}
    \tightlist
    \item
      Paradox: among rich industrial democracies of North Atlantic, the
      one that on paper has most progressive tax system is US

      \begin{itemize}
      \tightlist
      \item
        Relies on taxing personal income
      \item
        European social democracies based on regressive/indirect
        taxation of consumption (VAT)
      \end{itemize}
    \item
      In the short run, what matters most for overall redistributive
      effect is \emph{aggregate} tax take and how it is spent
    \item
      Progressive taxation has marginal effect on inequality, but
      politicians' commitment to progressive taxation at election time
      becomes device to show who's side they are on
    \end{itemize}
  \item
    Role for progressive redistributive taxation

    \begin{itemize}
    \tightlist
    \item
      Two targets: hierarchy of standards of living, accumulation of
      economic power
    \item
      Progressive income tax hits neither squarely
    \item
      Which instrument hits first target? Individualized consumption tax

      \begin{itemize}
      \tightlist
      \item
        Nicholas Kaldor: The Expenditure Tax
      \item
        Take aggregate income (labor and capital) and subtract invested
        savings; tax this difference on steeply progressive slope
      \end{itemize}
    \item
      Second target: tax system reaches more bluntly

      \begin{itemize}
      \tightlist
      \item
        Tax hereditary transfer of property
      \end{itemize}
    \end{itemize}
  \end{enumerate}
\item
  Central theme: equality without empowerment is a failed program
\end{itemize}

\hypertarget{the-relation-between-finance-and-the-real-economy-as-a-context-for-reshaping-the-market-order}{%
\subsection{The relation between finance and the real economy as a
context for reshaping the market
order}\label{the-relation-between-finance-and-the-real-economy-as-a-context-for-reshaping-the-market-order}}

\textbf{Unger}

\begin{itemize}
\tightlist
\item
  Three themes today:

  \begin{enumerate}
  \def\labelenumi{\arabic{enumi}.}
  \tightlist
  \item
    Present coronavirus crisis as a transformative opportunity
  \item
    Relation of finance to real economy, financial instability to
    economic breakdown
  \item
    Develop second discussion in an application to the nature, causes,
    and consequences of 2008 crisis
  \end{enumerate}
\item
  Two major themes of class have been: (1) alternative institutional
  forms that market economy can take and (2) evolution of productive
  apparatus of society, in particular, the most advanced practice of
  production and its relation to rest of economy

  \begin{itemize}
  \tightlist
  \item
    Frequently referred to idea of a ``war economy without a war''
  \item
    History has handed us an astonishing and unwanted example of this
    problem
  \item
    We have a crisis that in some respects evokes idea of war economy --
    many political leaders referred to this concept
  \item
    It is (should be) a war economy in the emergency production of
    medical equipment

    \begin{itemize}
    \tightlist
    \item
      State is source of both demand and supply
    \end{itemize}
  \item
    But in a larger sense, this is the inverse of a war economy

    \begin{itemize}
    \tightlist
    \item
      A radical breakdown in supply and demand where the major part of
      labor force is sent home
    \end{itemize}
  \end{itemize}
\item
  Transformative response to this problem in 4 areas:

  \begin{enumerate}
  \def\labelenumi{\arabic{enumi}.}
  \tightlist
  \item
    Income of workers
  \item
    Skills of labor force
  \item
    Revenue of businesses
  \item
    Organization of emergency production
  \end{enumerate}
\item
  Propose thought experiment: consider what a progressive political
  force would do in response to this emergency if it had the
  institutional and productivist focus that almost all existing
  progressive forces lack

  \begin{itemize}
  \tightlist
  \item
    Conception of what could happen if we had the right ideas and the
    right will
  \end{itemize}
\item
  Preliminary: who will pay for it? Initiatives require a large amount
  of money

  \begin{itemize}
  \tightlist
  \item
    If ever there were a circumstance in which state has no narrow
    public finance constraints, this is it
  \item
    All of this spending would represent replacement income and revenue
  \item
    So a great deal could be spent without provoking inflationary spiral
  \item
    State could finance initiatives by creating money -- increasing or
    monetizing public debt
  \end{itemize}
\item
  Income of workers: state has to guarantee large part of wage that
  workers no longer earn

  \begin{itemize}
  \tightlist
  \item
    Same principle must apply to self-employed or semi-employed workers
  \item
    Medium to long term: opportunity to experiment with insurance
    against ``black swan'' risks -- radicalize, generalize principle of
    social security
  \end{itemize}
\item
  Skills of labor force: opportunity to use crisis to increase skills of
  labor force

  \begin{itemize}
  \tightlist
  \item
    Network with each other within and outside firms
  \item
    Long term -- not just boost to productivity, increase in extent to
    which workers are independent from firms
  \end{itemize}
\item
  Business revenue: governments around the world propose to guarantee
  revenue

  \begin{itemize}
  \tightlist
  \item
    Should not be uncompensated -- demand reform to executive
    remuneration, increase in workers' rights
  \end{itemize}
\item
  Emergency production: specific corner of this problem in which example
  of war economy is most apposite

  \begin{itemize}
  \tightlist
  \item
    Governor of NY: propose medical supply chain be nationalized,
    President resisted: ``nationalization is a bad concept''
  \item
    War Production Act 1950 gives wide powers to federal government to
    direct businesses to produce what needs to be produced

    \begin{itemize}
    \tightlist
    \item
      Point of departure for innovative institutional engineering
    \item
      Not just centralized direction of production, but development of
      public-private partnerships to work in areas that don't correspond
      to short-term demands, but to long-term strategic imperatives
    \end{itemize}
  \end{itemize}
\item
  No insuperable technical, financial, economic or legal obstacle

  \begin{itemize}
  \tightlist
  \item
    Obstacles are fundamentally political and ideological -- ideas and
    will
  \end{itemize}
\end{itemize}

\textbf{Rodrik}

\begin{itemize}
\tightlist
\item
  Agree that crisis is not one that poses significant
  technical/intellectual challenges

  \begin{itemize}
  \tightlist
  \item
    Puzzle is why so much of the world is failing at meeting those
    challenges
  \item
    Many of ideas have not percolated up
  \item
    Massive problem of social cooperation and coordination, not just
    ideas and will
  \end{itemize}
\item
  Alternative thought experiment: imagine US had taken threat seriously
  -- testing kits employed, federal government had invested in
  widespread testing, would have enabled us to tell who was infected →
  enabled us to quarantine those who are infected

  \begin{itemize}
  \tightlist
  \item
    Only stay home 2 weeks → 1.5\% GDP lost
  \item
    Instead, we have supply shock with spillover into demand side
  \item
    Now estimate Great Depression levels of unemployment
  \item
    South Korea taken an approach closer to the one here

    \begin{itemize}
    \tightlist
    \item
      Deployed widespread testing and social distancing very rapidly
    \item
      Not an authoritarian state, so this is possible in democracy
    \end{itemize}
  \end{itemize}
\item
  Traditional Keynesian approach won't work -- we have a supply shock
  (ask people to stay home), so significant fall in GDP is inevitable
  (part of needed public health response)

  \begin{itemize}
  \tightlist
  \item
    What we can do is support income
  \item
    Britain/Germany: firms being paid to keep workers employed
  \end{itemize}
\item
  Uncertainty about crisis → supply shock creating induced demand shock

  \begin{itemize}
  \tightlist
  \item
    Even if paid 80\% of salary, uncertainty about future → consumers
    don't want to spend a lot money
  \item
    So, needs to be significant Keynesian action to increase aggregate
    demand
  \item
    Particularly with respect to medical supplies

    \begin{itemize}
    \tightlist
    \item
      Hard to understand why federal government is not directly
      requisitioning medical materials from suppliers
    \item
      Letting states compete with each other
    \end{itemize}
  \end{itemize}
\item
  Failure of global cooperation

  \begin{itemize}
  \tightlist
  \item
    WHO could have played much bigger role
  \item
    Arguably greater need for cooperation in public health than in
    economics (IMF, OECD)
  \end{itemize}
\item
  Failure of many political systems to produce leadership and response
  makes DR skeptical that we can envisage bridge to productivist agenda

  \begin{itemize}
  \tightlist
  \item
    Optimistic side: no longer take for granted social/economic
    arrangements, role of the state, prevailing nature of ideas --
    opening for the kinds of things we've been discussing
  \end{itemize}
\end{itemize}

\textbf{Unger}

\begin{itemize}
\tightlist
\item
  Promise of interaction between bold governmental initiative and
  cooperation/social organization
\item
  Disposition to cooperate is influenced by culture

  \begin{itemize}
  \tightlist
  \item
    But not just a constant
  \end{itemize}
\item
  Emergency production of needed medical equipment: cooperation becomes
  essential

  \begin{itemize}
  \tightlist
  \item
    Ex: Automobile factories don't immediately know how to produce
    ventilators
  \end{itemize}
\item
  Capacious view: not just the state acting unilaterally, rather
  creating opportunities for cooperative breakthroughs
\end{itemize}

\textbf{Rodrik}

\begin{itemize}
\tightlist
\item
  Question of how we're going to pay for this doesn't arise

  \begin{itemize}
  \tightlist
  \item
    Low interest rate environment already
  \end{itemize}
\item
  Supply side crisis might generate inflationary pressure, but we
  haven't seen that

  \begin{itemize}
  \tightlist
  \item
    Induced demand side effects are playing a role in dampening
    inflation expectations
  \item
    Not a time to be fiscally conservative
  \end{itemize}
\item
  Fair amount of consensus:

  \begin{itemize}
  \tightlist
  \item
    Income support, support demand, invest in medical supply chain,
    increase capacity at hospitals, backstop financial system (happening
    through ECB and Fed), work with small and medium businesses to
    ensure they have resources
  \item
    Much of debate is about practicalities of that
  \end{itemize}
\item
  Can we just print money?
\item
  \textbf{Rodrik:} As long as you have a use for it, as long as it
  doesn't increase inflation

  \begin{itemize}
  \tightlist
  \item
    Inflation is the sign you're running into a resource constraint
  \item
    As long as inflationary expectations and inflation are low, that's a
    sign you can increase supply of money (``helicopter money'')
  \item
    Not a big difference between traditional economics and MMT on this
    front
  \end{itemize}
\item
  \textbf{Unger:} Distinction between two levels

  \begin{itemize}
  \tightlist
  \item
    One level: money created by state is being used to replace personal
    income/business revenue

    \begin{itemize}
    \tightlist
    \item
      In principle, you can go a long way without provoking inflation
    \end{itemize}
  \item
    Second level: inflation becomes functional equivalent of a tax

    \begin{itemize}
    \tightlist
    \item
      Intervening in distributive conflict with advantages and
      disadvantages
    \item
      Historically, one way states have managed distributive conflict
    \end{itemize}
  \end{itemize}
\item
  \textbf{Rodrik:} Even in first level, loss in income is happening
  because people are not working

  \begin{itemize}
  \tightlist
  \item
    If we give them income, it's income that has no counterpart in real
    production
  \item
    Immediately find ourselves in second phase -- we cannot give them
    real income, only can give income by taking purchasing power away
    from someone else
  \item
    But this is not the situation we're in because no increase in
    inflationary expectations
  \item
    Loss in income to workers generated by loss of expenditures
    (reduction of demand) -- but we've generated supply shock by asking
    people to stay at home

    \begin{itemize}
    \tightlist
    \item
      So there's an element that can't be compensated by giving people
      more money
    \item
      Unless you find a way to increase production
    \end{itemize}
  \end{itemize}
\item
  \textbf{Unger:} Continuing dispute with DR, ``if it's already hard to
  reach the minimum, why are you asking for the maximum?'' → need to
  have higher transformative ambition to deal with basics
\item
  \textbf{Rodrik:} Interpret this as: ``Why make it difficult if you can
  make it impossible?''
\item
  \textbf{Unger:} Need interaction between state initiative and
  mobilization of organized civil society

  \begin{itemize}
  \tightlist
  \item
    Historical reality is that interaction cannot happen in an
    atmosphere of programmatic minimalism
  \item
    Only happens when there's a transformative campaign
  \end{itemize}
\end{itemize}

\textbf{Rodrik}

\begin{itemize}
\tightlist
\item
  Share of financial industry profits as total business, share of
  finance in US GDP, stock market trading volume over GDP, all up
\item
  Prior to global financial crisis, extraordinary scale: Ireland,
  Iceland (acting as hedge fund with stocks of financial assets and
  liabilities 10x GDP)
\item
  One would expect that some good would come out of financialization

  \begin{itemize}
  \tightlist
  \item
    Narrative behind financial liberalization, deregulation,
    globalization is that they produce benefits

    \begin{itemize}
    \tightlist
    \item
      Within economy: diversification of risk, distribution of risk from
      those who do not have ability to bear it to those who do

      \begin{itemize}
      \tightlist
      \item
        Enable access to homeownership to larger segments of population
      \end{itemize}
    \item
      Globally: transfer financial resources from rich countries to poor
      countries
    \end{itemize}
  \item
    World would experience boom in private investment and growth while
    experiencing reduction in overall volatility

    \begin{itemize}
    \tightlist
    \item
      Those expectations have not been borne out
    \item
      No increase in real resources devoted to investment in developing
      or advanced countries

      \begin{itemize}
      \tightlist
      \item
        If anything, private investment has stagnated
      \end{itemize}
    \end{itemize}
  \end{itemize}
\item
  Left with deep disconnect between finance and real economy

  \begin{itemize}
  \tightlist
  \item
    At best, a source of consumption support for parts of the middle
    class
  \item
    Spurred homeownership, but in a way that created systemic risk and
    financial distress
  \item
    Exception: moments of crisis, central banks act as lenders of last
    resort to commercial banking sector, parts of real economy

    \begin{itemize}
    \tightlist
    \item
      This morning: Fed preparing credit facility for lending to small
      and medium sized enterprise
    \end{itemize}
  \end{itemize}
\item
  ``Smart'' development banking

  \begin{itemize}
  \tightlist
  \item
    Out of fashion -- many countries closed down
  \item
    Finance long-term investment in environments where not available
    through normal commercial means
  \item
    Complemented with another role: intelligence gathering, where are
    there opportunities for new investments?

    \begin{itemize}
    \tightlist
    \item
      Tightly connected to productive opportunities in private sector
    \item
      Make intelligence available to the state and other parts of
      private sector
    \end{itemize}
  \end{itemize}
\item
  Public venture funds

  \begin{itemize}
  \tightlist
  \item
    Governments can borrow at very low interest rates → set up
    quasi-independent venture funds, managed independently, take
    ownership shares in startups or firms investing in new technologies
  \item
    Direct investment towards employment-enhancing or green technologies
  \end{itemize}
\end{itemize}

\textbf{Unger}

4 sets of comments:

\begin{itemize}
\tightlist
\item
  Locate discussion of finance in general idea of economic instability

  \begin{itemize}
  \tightlist
  \item
    2 fundamental causes of instability

    \begin{enumerate}
    \def\labelenumi{\arabic{enumi}.}
    \tightlist
    \item
      Absence of spontaneous/automatic correspondence between supply and
      demand

      \begin{itemize}
      \tightlist
      \item
        Keynes concerned with one version -- violation of Say's Law,
        aggravated by downward rigidity of wage
      \item
        But there are many ways supply and demand can fail to be
        connected
      \end{itemize}
    \item
      Troubled relation of finance to real economy

      \begin{itemize}
      \tightlist
      \item
        Keynes: came out of psychological tradition of English political
        economy
      \item
        Liquidity of money balances allow them to serve as plaything of
        waves of elation, despondency, greed, and fear
      \item
        But this is a limited and inadequate understanding of a more
        general phenomenon
      \end{itemize}
    \end{enumerate}
  \end{itemize}
\item
  Orthodox way of thinking about finance, and what's missing from it

  \begin{itemize}
  \tightlist
  \item
    Orthodoxy: If there is a problem with finance, originates in failure
    of competition in capital market (such as asymmetry of information)
    or in failure of regulatory response to localized market failure

    \begin{itemize}
    \tightlist
    \item
      Under this view, makes no sense to say finance is more/less
      closely related to productive agenda of society
    \item
      This selective blindness is an example of the false assumption
      that market economy has single natural, necessary form
    \item
      When Americans in first half of 19th century disbanded national
      bank and created most decentralized system of credit that had ever
      existed, they were creating institutional arrangements that
      tightened relationship of finance to real economy, not simply
      regulating
    \end{itemize}
  \end{itemize}
\item
  Conventional way of thinking cannot elucidate some things:

  \begin{itemize}
  \tightlist
  \item
    Under existing arrangements of market economy, production system is
    largely self-financed on basis of retained, reinvested private
    earnings

    \begin{itemize}
    \tightlist
    \item
      What is the point of money in banks and stock markets?
    \item
      Vast amount of financial activity has oblique relation to
      production
    \end{itemize}
  \item
    Under these arrangements, finance is relatively indifferent to real
    economy in good times, destructive in bad times

    \begin{itemize}
    \tightlist
    \item
      When things are going well, productive system largely finances
      itself
    \item
      In times of trouble, distress spills over into real economy
    \end{itemize}
  \item
    In principle, most important responsibility of finance: fund
    productive agenda of society, fund creation of new assets in new
    ways

    \begin{itemize}
    \tightlist
    \item
      Turns out to be miniscule part of financial activity
    \item
      Venture capital, even in U.S, where it is most established,
      represents less than 0.5\% of financial activity
    \end{itemize}
  \item
    Explanation: structure/arrangements that leave finance disconnected
    from production

    \begin{itemize}
    \tightlist
    \item
      Best way to make it less dangerous is to make it more useful
    \end{itemize}
  \end{itemize}
\item
  Programmatic horizon: What can we do?

  \begin{itemize}
  \tightlist
  \item
    Negatively: discourage or prohibit financial activities that have no
    relation to enhancement of production or productivity

    \begin{itemize}
    \tightlist
    \item
      Options, derivatives contracts increase liquidity in commodity
      markets but not clear they have a similar role in equity markets →
      turn into instrument of gambling
    \end{itemize}
  \item
    Positively: channel finance into productive investment

    \begin{itemize}
    \tightlist
    \item
      Smart investment banks, quasi-public venture funds are examples of
      this
    \item
      Pension systems could be placed in diversified portfolios of risk
      that would do the undone work of venture capital
    \end{itemize}
  \end{itemize}
\item
  Connect to crisis of 2008. Conventional response to crisis emphasized
  stimulus (Keynesianism) and re-regulation of finance

  \begin{itemize}
  \tightlist
  \item
    Causal background of crisis -- what was origin of problem?

    \begin{itemize}
    \tightlist
    \item
      Most fundamental origin: country stopped producing enough goods
      and services that rest of world wanted
    \end{itemize}
  \item
    Masked absence of productivist project with form of economic growth
    led by mass consumption

    \begin{itemize}
    \tightlist
    \item
      In principle, requires democratization of purchasing power -- but
      in late 20th century, violently regressive redistribution of
      wealth

      \begin{itemize}
      \tightlist
      \item
        How to reconcile cheap money and debt/credit with regressive
        wealth/income?
      \item
        Replace property-owning democracy with fake credit economy, made
        possible in part by overvaluation of housing stock as collateral
      \end{itemize}
    \item
      Enabled by structural imbalances in world economy -- debt and
      credit style of growth financed by export of Chinese financial and
      trade surpluses

      \begin{itemize}
      \tightlist
      \item
        Each country used structural imbalance to escape imperative of
        structural change
      \item
        Americans escaped need for productivist project and need to
        distribute assets to uplift production/democratize demand
      \item
        Chinese replaced imperative of deepening internal market with
        style of growth based on exports to deficit-ridden US
      \end{itemize}
    \end{itemize}
  \item
    Aggravated by regulatory approach (left untouched by response to
    2008): regulatory dualism

    \begin{itemize}
    \tightlist
    \item
      Densely regulated area of finance where public moved (ex: FDIC),
      thinly regulated area supposedly populated by professionals who
      could take care of themselves
    \item
      Everything that was prohibited in densely regulated area
      repackaged and done under different label in thinly regulated area
    \end{itemize}
  \item
    Response to crisis left whole causal background almost entirely
    untouched

    \begin{itemize}
    \tightlist
    \item
      Didn't develop productivist project, didn't disturb regulatory
      dualism
    \end{itemize}
  \end{itemize}
\end{itemize}

\textbf{Rodrik}

\begin{itemize}
\tightlist
\item
  Revisit housing markets pre-2008. The narrative behind the credit
  market bubble in housing finance was one that was on the consumption
  side of the economy -- not directly productivist -- but story meant to
  be in line with social objective of increasing homeownership

  \begin{itemize}
  \tightlist
  \item
    Who wouldn't want financial markets to serve homeownership?
  \item
    Why don't we introduce real competition into mortgage lending → take
    it out of hands out of monopoly of mainline banks, allow non-banks
    to make mortgage loans → create more access

    \begin{itemize}
    \tightlist
    \item
      Also, deregulate so intermediaries can offer creative, affordable
      mortgages
    \end{itemize}
  \item
    Worried about risk? Pool and package them into securities → farm
    them off to investors

    \begin{itemize}
    \tightlist
    \item
      Divvy up stream of payments into tranches, compensating holders of
      riskier ones, call on credit rating industries to certify that
      less risky ones are safe, then create derivatives to allow
      investors to purchase insurance (credit default swap)
    \end{itemize}
  \item
    Story is compelling -- financial innovation, increased competition,
    new intermediaries, new assets expanding benefits of middle-class
    homeownership
  \end{itemize}
\item
  Possible to see through tools of conventional economics that something
  was going to go wrong

  \begin{itemize}
  \tightlist
  \item
    Incentives create tremendous leverage, tendency for excessive risk
    taking not being internalized
  \item
    Cautionary note: ensure that financial innovation is disciplined by
    economics
  \end{itemize}
\end{itemize}

\textbf{Unger}

\begin{itemize}
\tightlist
\item
  One aspect of what happened in this story: mortgage market came out of
  New Deal, most powerful device for deepening of capital markets

  \begin{itemize}
  \tightlist
  \item
    Organized through ingenious institutions -- Freddie Mac, Fannie Mae
    (quasi-public)
  \end{itemize}
\item
  Late 20th century: securitization of mortgage market under the pretext
  of increasing competition

  \begin{itemize}
  \tightlist
  \item
    But became device of speculation disconnected from economic reality
  \item
    Layers of financial engineering -- transactions of real economy,
    rather than being subject of engineering, became simply the pretext
  \end{itemize}
\item
  Inequality increasing in US despite broad-based mass consumption

  \begin{itemize}
  \tightlist
  \item
    Inequality + cheap money → asset bubbles
  \end{itemize}
\item
  What's the regulatory response? 4 sets of programs

  \begin{enumerate}
  \def\labelenumi{\arabic{enumi}.}
  \tightlist
  \item
    Resurrection of New Deal policies and rules separating investment
    banks from regular banks
  \item
    New supervisory, liquidating authority in federal government (deal
    with too big to fail problem)
  \item
    International finance: new standards of capital adequacy
  \item
    New standards for consumer protection
  \end{enumerate}
\item
  None of these things dealt with regulatory dualism and separation from
  real economy → complete failure

  \begin{itemize}
  \tightlist
  \item
    Now, new crisis against background of old failure
  \end{itemize}
\item
  Every time we try regulatory dualism, the same thing happens: we
  repackage all the things that are prohibited in thickly regulated
  sector and do them in the thinly regulated one
\item
  Something like half of graduating class of Harvard College go into
  finance and management consulting

  \begin{itemize}
  \tightlist
  \item
    Pushing paper, gambling, rather than investing, making things, doing
    real things in the world
  \item
    Absorbing leviathan proportion of profits, perverting and
    misdirecting the intellectual energies of the nation
  \end{itemize}
\item
  Q: How do we reconcile nationalization with decentralized production?
\item
  \textbf{Rodrik:} Nationalization is a question about ownership.
  Mechanism through which it's carried out can be decentralized → no
  conflict between ownership and control rights and how the work is
  organized
\item
  \textbf{Unger:} Think again of the metaphor of wartime production:
  government says stop making cars, we need tanks → provides help, puts
  them in touch with people, but all it wants is the tanks
\item
  Need to invent the institutional machinery of competitive
  experimentalism. The public initiative should not be seen as a
  suppression of that
\item
  Logic of the market is the logic of an organized experimental anarchy
  -- we want more of that
\item
  Need initiative by the state to help us have more of it
\end{itemize}

\hypertarget{the-relation-between-labor-and-capital-as-a-context-for-reshaping-the-market-order}{%
\subsection{The relation between labor and capital as a context for
reshaping the market
order}\label{the-relation-between-labor-and-capital-as-a-context-for-reshaping-the-market-order}}

\textbf{Unger}

\begin{itemize}
\tightlist
\item
  Critique of neoclassical theory of firm leads to different options
  that are all unsatisfactory

  \begin{itemize}
  \tightlist
  \item
    Business roundtable (social responsibility) → place evaluation of
    ``social direction'' in hands of managerial elite

    \begin{itemize}
    \tightlist
    \item
      In practice: genuflect to political correctness of the day,
      reinforce whatever the current platitudes are. Why do we want
      this?
    \item
      Harvard Business School motto: ``Make profit with a conscience.''
      Would rather they just make profit and leave conscience to the
      public
    \end{itemize}
  \item
    Capital-intensive parts of economy: workers and employers have
    common interest against everyone else
  \item
    Collaboration model: kind of corporatism in which firms do
    everything together with the state

    \begin{itemize}
    \tightlist
    \item
      Normal point of departure: as a condition for receiving corporate
      charter, the corporation must obey whatever is established by law
    \item
      Law determines social direction and conditions of corporate
      activity
    \item
      To describe it as ``regulatory policy'' trivializes it
    \end{itemize}
  \end{itemize}
\end{itemize}

\textbf{Rodrik}

\begin{itemize}
\tightlist
\item
  Skeptical of social responsibility model, but two practical
  advantages:

  \begin{itemize}
  \tightlist
  \item
    Only one of four strategies that we've seen progress on

    \begin{itemize}
    \tightlist
    \item
      Large firms are doing something with practical effects
    \end{itemize}
  \item
    Rebecca Henderson at HBS one of the strongest proponents

    \begin{itemize}
    \tightlist
    \item
      She says this is not an alternative to the government or the law
    \item
      Makes them more likely to be willing participants in
      government-led approaches
    \end{itemize}
  \end{itemize}
\item
  When labor has sense of ownership, participation, good things follow:
  satisfaction, productivity, technology adoption that is more cognizant
  of interests of labor
\item
  Law cannot specifically say what wage rate gig workers doing MTurk in
  this industry in that region should get, other complicated issues
  about employment standards

  \begin{itemize}
  \tightlist
  \item
    How do those get determined?
  \end{itemize}
\end{itemize}

\textbf{Unger}

\begin{itemize}
\tightlist
\item
  Law can say: you can have flexible contracts, but you have to pay them
  what you would otherwise

  \begin{itemize}
  \tightlist
  \item
    Courts and jurists develop series of standards and criteria
  \item
    Law doesn't determine details, it sets a direction
  \end{itemize}
\item
  If we listen to business school, we always have the pietistic prevail
  over the transformative
\end{itemize}

\textbf{Rodrik}

\begin{itemize}
\tightlist
\item
  Need flexibility in policy that the law cannot provide
\end{itemize}

Q: Do we think legal institutions are less prone to capture? In U.S.,
laws like right-to-work being used against workers.

\textbf{Unger}

\begin{itemize}
\tightlist
\item
  That's the general problem of politics. Law is not part of the
  institutions
\item
  100\% of the institutions are manifest in legal detail; they come from
  struggle in society over the arrangements
\item
  We prefer this struggle over subcontracting part of democratic power
  to a group such as the managerial elite
\item
  If there's capture, the solution is more democracy
\item
  Can't hedge against politics. Politics is fate

  \begin{itemize}
  \tightlist
  \item
    Either you give the power to the people and recognize that they may
    make mistakes
  \item
    Or you give power to custodians
  \end{itemize}
\item
  In the US, American liberals tried to sneak program through federal
  judiciary

  \begin{itemize}
  \tightlist
  \item
    What always happens: the empire strikes back
  \end{itemize}
\item
  In terms of custodians, the managers are at the bottom of the moral
  hierarchy
\end{itemize}

\textbf{Rodrik}

\begin{itemize}
\tightlist
\item
  Corporate social responsibility is partly due to abdication of
  national governments
\item
  Firms might rather know what the rules are than have to voluntarily
  participate
\item
  We have some international examples, but most of the action needs to
  happen at the national level
\end{itemize}

\textbf{Unger}

\begin{itemize}
\tightlist
\item
  Emphasize the labor force and the nature of free labor in the
  contemporary economy, rather than the firm and its responsibility
\item
  2 connections to general themes:

  \begin{enumerate}
  \def\labelenumi{\arabic{enumi}.}
  \tightlist
  \item
    Linked to insular/inclusive knowledge economy

    \begin{itemize}
    \tightlist
    \item
      Technologically-provoked unemployment (automation discourse)

      \begin{itemize}
      \tightlist
      \item
        Take position that technological evolution is underdetermined →
        choice between technology replacing vs.~augmenting labor
      \end{itemize}
    \item
      Historically, organization of labor within firms preceded by
      ``putting-out system''

      \begin{itemize}
      \tightlist
      \item
        Now we see a new putting-out system
      \item
        What we consider the ``natural'' way of organizing labor might
        turn out to be just an interlude between two putting-out systems
      \item
        Increasing part of labor force consigned to precarious
        employment -- can't allow ``flexibility'' to serve as pretext
        for precarity
      \end{itemize}
    \end{itemize}
  \item
    Alternative institutional forms of market economy:

    \begin{itemize}
    \tightlist
    \item
      Best thing is not to regulate, or compensate after the fact but to
      change it, in 3 domains:

      \begin{enumerate}
      \def\labelenumii{\arabic{enumii}.}
      \tightlist
      \item
        Organization of production, relation of backwards/advanced parts
      \item
        Relation of finance to economy
      \item
        Relation of labor to capital
      \end{enumerate}
    \end{itemize}
  \end{enumerate}
\end{itemize}

\emph{Short-term program}

\begin{itemize}
\tightlist
\item
  Proximate objective: increase chances that labor will be organized in
  a way that supports broad-based rise in productivity
\item
  Ulterior objective: enhance economic freedom, agency
\item
  Marx: most important feature of capitalism is that labor can be bought
  and sold

  \begin{itemize}
  \tightlist
  \item
    Is this the way things have to be?
  \end{itemize}
\item
  Establish policies and arrangements that increase likelihood that
  technology evolves in a way that augments rather than replaces labor.
  3 categories of actions:

  \begin{enumerate}
  \def\labelenumi{\arabic{enumi}.}
  \tightlist
  \item
    Ways in which state can exemplify, encourage, mandate productive
    solutions that augment labor

    \begin{itemize}
    \tightlist
    \item
      Direct government action in areas such as defense technology
    \item
      Fiscal and regulatory policy
    \end{itemize}
  \item
    Industrial policy: attempt to uplift retrograde firms to bring them
    closer to frontier; assist autonomous economic agents → transform
    into technologically-equipped artisans
  \item
    Empower labor (by law) to have more access to advanced part of
    production, more power within advanced part of production
  \end{enumerate}
\item
  Legal regime of organization of labor. 2 main sectors:

  \begin{enumerate}
  \def\labelenumi{\arabic{enumi}.}
  \tightlist
  \item
    Parts of economy with stable employment

    \begin{itemize}
    \tightlist
    \item
      What can or should the labor law regime be? 2 regimes:

      \begin{enumerate}
      \def\labelenumii{\arabic{enumii}.}
      \tightlist
      \item
        Collective bargaining that prevails in North Atlantic, developed
        to redress radical inequality of bargaining power in employment
        context. 2 principles: (1) decision to unionize is voluntary,
        (2) unions are independent from state
      \item
        Corporatist regime that predominates in Latin America.
        Originally developed in fascist Italy, adopted by leftist
        regimes in Latin America. 2 principles: (1) everyone
        automatically unionized, (2) union structure under direction of
        state, typically through Ministry of Labor
      \end{enumerate}
    \item
      Despite suspect origins, corporatist regime has 3 advantages:

      \begin{enumerate}
      \def\labelenumii{\arabic{enumii}.}
      \tightlist
      \item
        Unionization is automatic. A ``gift of the law.'' Struggle of
        workers is not whether to unionize, but what to do with union
        power
      \item
        Solidaristic tilt: less likely that union system reflects and
        reinforces underlying inequalities of production system
      \item
        Favors claims that are more than economistic (i.e.~wages and
        benefits) -- has an institutional and political dynamic
      \end{enumerate}
    \item
      Unacceptable vice: subordination of union movement to the state
    \item
      Solution? Hybrid regime:

      \begin{itemize}
      \tightlist
      \item
        From corporatist regime, take automatic unionization
      \item
        From collective bargaining, take freedom of union from the state
      \end{itemize}
    \end{itemize}
  \item
    Unstable/precarious employment. 2 discourses:

    \begin{enumerate}
    \def\labelenumii{\arabic{enumii}.}
    \tightlist
    \item
      Traditional labor discourse predicated on collective
      bargaining/corporatist regimes, wants to outlaw unstable
      employment as circumvention of labor laws. 2 defects:

      \begin{enumerate}
      \def\labelenumiii{\arabic{enumiii}.}
      \tightlist
      \item
        Can't turn back the clock and suppress new practices of
        production by decree
      \item
        Serves interest of organized minority against disorganized
        majority
      \end{enumerate}
    \item
      Neoliberal discourse preaches radical flexibility

      \begin{itemize}
      \tightlist
      \item
        Erode rights of workers and throw them into radical insecurity
      \end{itemize}
    \end{enumerate}
  \end{enumerate}

  \begin{itemize}
  \tightlist
  \item
    Should distinguish flexibility (unavoidable or desirable) from
    radical economic insecurity
  \item
    Need a new legal regime for unstable employment

    \begin{itemize}
    \tightlist
    \item
      These workers should be organized and represented
    \item
      Legal principle of price neutrality: unstable employment
      remunerated at level at which they would be remunerated in stable
      employment
    \item
      One point of departure: Scandinavian system of labor rights and
      social rights that are universally portable
    \end{itemize}
  \end{itemize}
\end{itemize}

\emph{Long-term program}

\begin{itemize}
\tightlist
\item
  Economic freedom becomes a central aim. 3 aspirations:

  \begin{enumerate}
  \def\labelenumi{\arabic{enumi}.}
  \tightlist
  \item
    No human should be condemned to do work that can be done by a
    machine

    \begin{itemize}
    \tightlist
    \item
      Reserve our time for the not yet repeatable
    \item
      Machines don't have imagination
    \item
      Create condition for our partnership with machines
    \item
      This is the opposite of traditional forms of production: worker
      worked as if he were a machine

      \begin{itemize}
      \tightlist
      \item
        We want the opposite, but it's unlikely as long as wage labor
        becomes predominant form of free work
      \end{itemize}
    \end{itemize}
  \item
    Character of free labor. 3 main forms: wage labor, self-employment,
    cooperation

    \begin{itemize}
    \tightlist
    \item
      Liberals and socialists up to middle of the 19th century thought
      that economically dependent wage labor is a inferior and
      transitive form of free labor that would give way over time to
      self-employment and cooperation

      \begin{itemize}
      \tightlist
      \item
        Everyone believed this, not just Marx. Lincoln, John Stuart Mill
      \item
        End of 19th century → predominance of wage labor came to seem
        natural and necessary
      \item
        How do we implement this idea in contemporary economy?
      \item
        Idea ran into problem of scale in 19th century
      \end{itemize}
    \end{itemize}
  \item
    Need innovations in property regime

    \begin{itemize}
    \tightlist
    \item
      Develop range of ways in which economic agents can have
      decentralized access to productive resources and opportunities
    \item
      Property is a bundle of powers that can be disassembled,
      components vested in different stakeholders → superimposed claims
      on same productive resource

      \begin{itemize}
      \tightlist
      \item
        Ex: productive resources vested in decentralized funds, use
        auctioned off to those groups who can offer highest rate of
        return (temporarily)
      \item
        Alternative systems of property and contract coexisting
        experimentally within same market order
      \end{itemize}
    \end{itemize}
  \end{enumerate}
\end{itemize}

\emph{Philosophical remark}

\begin{itemize}
\tightlist
\item
  The premise of this set of ideas is that we can aspire to have freedom
  in the economy, not just freedom from the economy
\item
  Marx and Keynes believed we were on the verge of overcoming scarcity,
  and in the world beyond we could escape hateful burden of work and
  enjoy ``private sublimities''

  \begin{itemize}
  \tightlist
  \item
    Both of these are false. We are not about to overcome scarcity
  \item
    Also, not true that work is just a hateful burden

    \begin{itemize}
    \tightlist
    \item
      Under the knowledge economy, if we heighten trust and make work
      approximate the activity of imagining and producing, work can
      cease to be a hateful burden. But for whom?
    \end{itemize}
  \end{itemize}
\end{itemize}

\textbf{Rodrik}

\begin{itemize}
\tightlist
\item
  Agree on wage neutrality: unstable workers → same conditions as in
  organized labor force. There should not be a dualism in structure of
  employment

  \begin{itemize}
  \tightlist
  \item
    In the taxi industry, avoid ``Uberization'' of workforce
  \end{itemize}
\end{itemize}

\textbf{Unger}

\begin{itemize}
\tightlist
\item
  Corporatist labor law regimes: Unions negotiate on behalf of large
  categories of workers sector by sector
\end{itemize}

\textbf{Rodrik}

\begin{itemize}
\tightlist
\item
  Advantage of corporatism: universality
\item
  Labor power, even in capital intensive parts of production (Amazons
  and Googles), might have valuable spillovers? RMU disagrees
\item
  New technologies can augment labor directly or indirectly:

  \begin{itemize}
  \tightlist
  \item
    Technology that gives nurses skills to do what doctors do is a
    direct augmentation of nurses' labor
  \item
    Alternative: increase number of tasks that need to be performed

    \begin{itemize}
    \tightlist
    \item
      New technology enables teachers to discern learning styles of
      different students and provide customized education in a highly
      differentiated way
    \item
      Increases the tasks that teachers can do -- indirect augmentation
      of labor
    \end{itemize}
  \item
    How do we change incentives to do both of these things?
  \end{itemize}
\end{itemize}

\textbf{Unger}

\begin{itemize}
\tightlist
\item
  Up to now, thought of this as race between human and machine
  (i.e.~who's better at playing chess?)

  \begin{itemize}
  \tightlist
  \item
    In certain respects, machines are more powerful
  \item
    But we have a different hierarchy in the direction of imagination
  \item
    Subsume actual under range of transformative variation
  \item
    We can discover before we've made sense of something, in violation
    of our own methods and presuppositions, and then develop methods to
    make sense of them retrospectively
  \item
    Enhances hope of partnership between human and machine
  \end{itemize}
\item
  Creation of breakthroughs depends on sifting through alternatives:
  experimentation

  \begin{itemize}
  \tightlist
  \item
    Humans have to do it through time. Experimentalism requires time,
    but it's our scarcest resource
  \item
    Machine does it through computational power, which can be almost
    instantaneous
  \item
    If machines can help us deal with our scarcest resource, time, they
    can be immense ally in our liberation
  \end{itemize}
\end{itemize}

\hypertarget{globalization-and-alternative-globalizations-the-international-counterpart-to-these-national-debates}{%
\section{Globalization and alternative globalizations: the international
counterpart to these national
debates}\label{globalization-and-alternative-globalizations-the-international-counterpart-to-these-national-debates}}

\textbf{Rodrik}

\begin{itemize}
\tightlist
\item
  William Jennings Bryan speech at DNC, 1896: ``You shall not crucify
  mankind upon a cross of gold''

  \begin{itemize}
  \tightlist
  \item
    Gold standard was target of late 19th century American populists
  \item
    Led by farmers angry at unification of commodity markets at moment
    of price deflation → prices of agricultural products coming down
  \item
    Relatively scarce gold supplies → nominal interest rates high, even
    higher real interest rate, given deflation
  \item
    This was a consequence of ``tight money'' -- no freedom to increase
    money supply, since it was linked to supply of gold
  \item
    Backlash against financiers and bankers in Northeast seen as
    proponents of these policies

    \begin{itemize}
    \tightlist
    \item
      Analog to today: globalization imposing strict constraints on
      conduct of economic policy at home which harmed sectors of economy
      (here, farmers) → political expression
    \end{itemize}
  \end{itemize}
\item
  Alternative globalizations

  \begin{itemize}
  \tightlist
  \item
    We take for granted a particular type of economic globalization:
    IMF, WTO, global supply chains
  \item
    All globalizations run on rules (explicit) and norms (internalized)
    -- don't spontaneously arise from technology

    \begin{itemize}
    \tightlist
    \item
      Majority of countries that have embraced financial globalization
      (open capital markets) because they think that's what good
      behavior is, not because they're following rules
    \item
      Who writes the rules? Whose preferences are privileged?
    \end{itemize}
  \item
    Could have globalization targeted at preventing/mitigating
    pandemics, with WHO at the center

    \begin{itemize}
    \tightlist
    \item
      Warning system, research budget for pandemics, prevention of
      export controls on medical equipment, regulated border
      closures\ldots{}
    \end{itemize}
  \item
    Another alternative: Globalization targeted at slowing down climate
    change

    \begin{itemize}
    \tightlist
    \item
      Globally binding emission quotas, research budget for renewable
      energy, financing for transition to green energy
    \end{itemize}
  \item
    Globalization focused on human development and empowerment
  \item
    Alternative forms of economic globalization, too

    \begin{itemize}
    \tightlist
    \item
      Which flows should be liberalized? Trade, finance, labor?
    \item
      Should rules reach behind borders? In which areas?

      \begin{itemize}
      \tightlist
      \item
        Subsidies, industrial policies, emissions, tax regimes
      \end{itemize}
    \item
      Ex: Gold standard included aspiration to labor mobility, but not
      Bretton Woods or post-1990s hyperglobalization
    \item
      Ex: Gold standard and post-1990s regime included restraints on
      domestic policies, but not Bretton Woods
    \end{itemize}
  \end{itemize}
\item
  Tensions between democracy and economic globalization

  \begin{itemize}
  \tightlist
  \item
    Trilemma: choose two of national sovereignty, hyperglobalization,
    mass politics
  \item
    Hyperglobalization = absence of transaction costs on cross-border
    trade and finance

    \begin{itemize}
    \tightlist
    \item
      No restrictions on goods, services, assets at the border
    \item
      Harmonize monetary, legal, regulatory regimes
    \item
      Credibly pre-commit not to deviate from these regimes
    \end{itemize}
  \item
    Differences in national institutional arrangements create
    cross-border transaction costs and impede globalization
  \item
    Hyperglobalization runs on the logic of arbitrage → undermines
    differences in regulations and social models

    \begin{itemize}
    \tightlist
    \item
      More advanced labor standards? Firms will shift production to
      jurisdictions with lower labor standards
    \item
      More capital market regulations? Banks go to locations where
      regulations are lax and export products
    \item
      More progressive income tax? Wealthy individuals and corporations
      will move to lower-tax jurisdiction but do business in your market
    \end{itemize}
  \item
    Tensions are more manageable when economic integration is limited
    (GATT) or regulatory differences are small (US states)
  \item
    What kind of a state does hyperglobalization require?

    \begin{itemize}
    \tightlist
    \item
      Provide property rights, contract enforcement, monetary stability
    \item
      No costs on free flow of goods or capital

      \begin{itemize}
      \tightlist
      \item
        Sacrifice other objectives to this
      \end{itemize}
    \item
      Problematic for democratic regimes, since we want the possibility
      of divergent policies/institutional arrangements
    \end{itemize}
  \item
    Democratic delegation: democratic rule can be constrained without
    creating undemocratic outcomes

    \begin{itemize}
    \tightlist
    \item
      Limit power of special interests
    \item
      Enhance quality of democratic deliberation
    \item
      Principle applies to international commitments, too
    \end{itemize}
  \item
    But there's a difference between logic of hyperglobalization
    (justify any rules that restrict domestic autonomy to minimize
    transaction costs) and democracy-enhancing globalization (impose
    mostly procedural norms that enhance deliberation)
  \item
    Gold standard: hyperglobalization and national sovereignty, seems
    incompatible with mass politics

    \begin{itemize}
    \tightlist
    \item
      Rules too strict for a representative system to maintain?
    \end{itemize}
  \item
    Bretton Woods: maximize democratic legitimacy at home -- Keynesian
    macro policies + welfare state + economic restructuring

    \begin{itemize}
    \tightlist
    \item
      Some sectors left out of international agreements
    \item
      When international trade threatened to overwhelm social bargains
      (1970s: exports from newly industrialized countries affect
      low-income sectors in advanced economics) → carve out exceptions
      to give advanced countries room
    \item
      Explicitly incomplete globalization
    \end{itemize}
  \item
    Hyperglobalization: Post-1990s, WTO + financial globalization

    \begin{itemize}
    \tightlist
    \item
      Due to failure of legitimacy where rules went too far (trade) --
      too far from democratic deliberation
    \item
      And failure of regulation where they didn't go far enough
      (finance)
    \end{itemize}
  \item
    Another alternative: ``global governance,'' do away with national
    sovereignty entirely

    \begin{itemize}
    \tightlist
    \item
      ``Quasi-federalism'' at the global level
    \item
      Ideal of the European Union
    \end{itemize}
  \end{itemize}
\item
  Outlines of a desirable form of economic globalization

  \begin{itemize}
  \tightlist
  \item
    Benefits to all rather than a few
  \item
    Enforce rules for global public goods
  \item
    Leaves space for institutional diversity across nations
  \end{itemize}
\end{itemize}

\textbf{Unger}

\begin{itemize}
\tightlist
\item
  Application of idea of hyperglobalization to labor mobility: seems
  like what DR and international technocracy mean by
  ``hyperglobalization'' is not meant to include radical advance in
  mobility of labor
\item
  Difference between mobility of capital/goods and labor is very
  significant
\end{itemize}

\textbf{Rodrik}

\begin{itemize}
\tightlist
\item
  Yes, post-1990s globalization has not included labor markets
\item
  But don't want to simply extend hyperglobalization to labor
\item
  Want more balanced tradeoff between pushing for gains from trade and
  maintaining institutional diversity

  \begin{itemize}
  \tightlist
  \item
    In the case of labor, would require us to move further towards
    globalization
  \item
    But for goods and capital, move back
  \end{itemize}
\end{itemize}

\textbf{Unger}

\begin{itemize}
\tightlist
\item
  One direction: there should be a radically different treatment of
  goods/capital and labor
\item
  Other direction: they should achieve freedom together in small,
  cumulative steps
\item
  This radical difference in direction is concealed under label
  ``hyperglobalization''
\end{itemize}

\textbf{Rodrik}

\begin{itemize}
\tightlist
\item
  Is the ultimate goal complete mobility? That would be
  hyperglobalization
\item
  My view: long-term tradeoff between gains from trade and gains from
  institutional diversity

  \begin{itemize}
  \tightlist
  \item
    This always establishes a limit to how much globalization we can
    have
  \end{itemize}
\end{itemize}

\textbf{Unger}

\begin{itemize}
\tightlist
\item
  Movement of things and money is sometimes helpful, sometimes harmful
\item
  Movement of people is sacrosanct: significance transcends the merely
  economic realm

  \begin{itemize}
  \tightlist
  \item
    But cannot be established instantaneously, radically
  \end{itemize}
\item
  Can't put this on a spectrum of more or less globalization
\item
  Radical capital mobility is dangerous: suppresses possibility of
  national development
\item
  Paradox in mainline economic theory: division of world into sovereign
  states is an accident, an embarrassment -- why divide the world at
  all? Just creates transaction costs

  \begin{itemize}
  \tightlist
  \item
    Integrated world equilibrium is just a fantasy
  \item
    Division of world into states is basis of trade theory
  \item
    Why is the world divided? Because we value difference -- for moral
    and political reasons, as well as economic benefits
  \end{itemize}
\end{itemize}

\textbf{Rodrik}

\begin{itemize}
\tightlist
\item
  Challenge idea that economics has no explanation for division of rules
  in sovereign units
\item
  Local public goods: different communities have different preferences →
  want different types of institutional arrangements
\item
  Idea of gains from trade rooted in comparative advantage; this theory
  applies to why we want different nation-states
\item
  But this theory doesn't explain why we have existing structure of
  nation-states
\end{itemize}

\textbf{Unger}

\begin{itemize}
\tightlist
\item
  Local public goods are a thin, impoverished basis for understanding
  value of nations
\item
  More fundamental basis: fecundity of method of competitive selection
  depends on richness of material from which method selects (analog to
  Darwin)
\item
  Not a provincial accident of people having different preferences in
  different places
\item
  Difference has to be \emph{created} -- we don't want just the
  difference that arises spontaneously
\item
  Another point: an assumption of established discourse about
  hyperglobalization is that more openness requires more harmonization
  (legal and institutional similarity) and fewer transaction costs

  \begin{itemize}
  \tightlist
  \item
    This is a fundamental mistake
  \item
    History of economics and law show that many earlier experiments in
    globalization have pushed extent to which openness and trade is
    compatible with underlying legal and institutional difference
  \item
    GATT characterized by legal and institutional minimalism

    \begin{itemize}
    \tightlist
    \item
      Prior history of same thing: lex mercatoria (merchant law) in
      European history
    \item
      Whole point of international commercial law was to reconcile
      increasing trade with underlying legal and institutional diversity
    \end{itemize}
  \item
    The impulse of this discourse of harmonization and exaggerated
    significance of transaction costs is to connect openness with legal
    and institutional similarity
  \end{itemize}
\end{itemize}

\textbf{Rodrik}

\begin{itemize}
\tightlist
\item
  How would we measure globalization/openness? Two different ways:

  \begin{itemize}
  \tightlist
  \item
    Quantities of flows. Lot of globalization because lot of trade, lot
    of investment crossing borders
  \item
    Size of the barriers: Are there costs to goods/services/people
    moving across borders
  \end{itemize}
\item
  These can move in different directions. Can have barriers that are
  relatively high (GATT regime, Middle Ages) -- but still have a lot of
  flows because existing barriers enable countries to manage countries
  better, creating more hospitable environment for the flows

  \begin{itemize}
  \tightlist
  \item
    Bretton Woods/GATT regime: expansion in trade. Not primarily because
    we reduced transaction costs; bulk of expansion because countries
    individually prospered → ended up trading more with each other
  \end{itemize}
\item
  Differences are definitionally barriers

  \begin{itemize}
  \tightlist
  \item
    People get around these barriers by costly evasion -- bypassing
    local sovereignty, etc.
  \end{itemize}
\end{itemize}

\textbf{Unger}

\begin{itemize}
\tightlist
\item
  Costs of differences are subsidiary to larger point: under established
  discourse which wants to associate openness with absence of
  difference, friends of difference are enemies of openness
\item
  There isn't a single spectrum of tradeoffs
\item
  To what extent can we reconcile openness with underlying institutional
  difference is an empirical question. We can't answer a priori
\end{itemize}

\textbf{Rodrik}

\begin{itemize}
\tightlist
\item
  Magnitude of transaction costs is empirical. But whether differences
  do indeed impose transaction costs can be answered a priori
\item
  Normatively appropriate benchmark is the barriers, not the volume
\end{itemize}

Q: Why shouldn't we make global governance a goal? Is harmonization of
human rights part of hyperglobalization?

\textbf{Rodrik}

\begin{itemize}
\tightlist
\item
  Some areas where it's easier to establish global norms, e.g.~human
  rights
\item
  Appropriately hyper-global regime (along with climate change, global
  health)
\item
  These are domains of global public goods
\item
  Not sure the same applies to labor. This economic obligation doesn't
  rise to the same level

  \begin{itemize}
  \tightlist
  \item
    Global set of rules creating completely free labor markets is maybe
    not desirable
  \end{itemize}
\end{itemize}

\textbf{Unger}

\begin{itemize}
\tightlist
\item
  We want world trade system to be based on free labor
\item
  One the one hand, different countries have to be able to compete on
  basis of different returns to labor

  \begin{itemize}
  \tightlist
  \item
    Don't want this to result in suppression of freedom of labor
  \end{itemize}
\end{itemize}

Q: Post-1990 system hasn't enhanced labor mobility, but not for lack of
trying (consensus on economic benefits of immigration).

\textbf{Rodrik}

\begin{itemize}
\tightlist
\item
  Business interests not successful at liberalizing immigration

  \begin{itemize}
  \tightlist
  \item
    Have not been as organized, though, compared to their efforts in IP,
    banking/financial regulation
  \item
    Businesses can feel direct benefits in, ex: liberalization of H1
    visas
  \item
    Not organized where gains are more diffuse -- low-skill labor
  \end{itemize}
\item
  To get things done, need to use ideas and narratives, too

  \begin{itemize}
  \tightlist
  \item
    ``Trade-related intellectual property rights'' (TRIPS) -- economists
    did not want this, but the narrative (from Big Pharma) gave their
    cause legitimacy
  \item
    We can provide counter-narratives
  \end{itemize}
\end{itemize}

Q: To what extent can environmental and health globalizations be
separated from economic integration?

\textbf{Rodrik}

\begin{itemize}
\tightlist
\item
  Could envisage world where every economy is independent (autarky) but
  climate is still a global public good
\item
  How do we impose rules on carbon taxes? Can just charge countries
  through body like UN. Doesn't mean they have to trade
\end{itemize}

\hypertarget{the-struggle-over-the-shape-of-the-market-order-and-the-contest-over-ways-of-thinking-the-role-of-economics-and-of-its-possible-reorientation}{%
\section{The struggle over the shape of the market order and the contest
over ways of thinking: the role of economics and of its possible
reorientation}\label{the-struggle-over-the-shape-of-the-market-order-and-the-contest-over-ways-of-thinking-the-role-of-economics-and-of-its-possible-reorientation}}

\textbf{Unger}

\begin{itemize}
\tightlist
\item
  Begin with background theme: alternative futures of economics itself.
  Where will we get the ideas we need to think about the problems we
  have been addressing?
\item
  This is where he and DR fundamentally disagree. DR sees nothing
  fundamentally wrong with established economics

  \begin{itemize}
  \tightlist
  \item
    RMU recognizes economics is the best established of social sciences
  \item
    Forces on us clarity, especially about tradeoffs and constraints
  \item
    But it's not enough
  \end{itemize}
\item
  Difficult to address nature of limitations/alternative directions
  because:

  \begin{itemize}
  \tightlist
  \item
    Confusion of relation of subject matter to method

    \begin{itemize}
    \tightlist
    \item
      Economics is not study of the economy; it's the study of a method
      pioneered by marginalist theoreticians of end of 19th century
    \item
      Study of economy by another method (e.g.: Weber) is not regarded
      as economics
    \item
      Study of something that has nothing to do with production and
      exchange might be regarded as economics because it deploys this
      method
    \end{itemize}
  \item
    Familiar criticisms (unrealistic assumptions about competition or
    rationality) are almost all uncomprehending → rightly dismissed

    \begin{itemize}
    \tightlist
    \item
      Arise from failure to understand character of method and its
      ambitions
    \item
      These misleading criticisms crowd out the criticisms we need
    \end{itemize}
  \item
    Attempted heresies during course of evolution of mainline economics

    \begin{itemize}
    \tightlist
    \item
      Dialectic of these largely failed heresies suggests we have
      everything we need by way of intellectual diversity
    \end{itemize}
  \item
    If we arrive at conclusion that economics in present state is
    insufficient, then we have no alternative readily available

    \begin{itemize}
    \tightlist
    \item
      Absence of alternative dissuades us from pursuing criticism and
      reconstruction
    \end{itemize}
  \end{itemize}
\item
  Place problem of economics in narrative of modern social thought

  \begin{itemize}
  \tightlist
  \item
    Classical European social theory was a theory of structure --
    structures that exist in history, how they change, laws that govern
    their operation and transformation
  \item
    Paradigm was Marx's critique of English political economy

    \begin{itemize}
    \tightlist
    \item
      English economists represent as universal laws what are actually
      regularities of particular economic regime: capitalism
    \item
      Regimes and structures are contingent and revisable

      \begin{itemize}
      \tightlist
      \item
        But this was circumscribed by classical social theory by a
        series of deterministic illusions
      \item
        Thought there was a closed list of regimes, each an indivisible
        system, with a script governing their foreordained succession
      \item
        Subsequent thought rebelled, but suppressed structural
        imagination altogether
      \end{itemize}
    \end{itemize}
  \item
    Each social science has severed link between insight into actual and
    imagination of adjacent possible

    \begin{itemize}
    \tightlist
    \item
      They present as explanation what is a retrospective
      rationalization of established arrangements
    \end{itemize}
  \item
    Problems of economics are a variant on fundamental theme:
    suppression of structural imagination
  \end{itemize}
\item
  Deal with main line of economics resulting from marginalist
  revolution, suggest alternative account of its limitations

  \begin{itemize}
  \tightlist
  \item
    Marginalists (Walras, Jevens, Menger) saw economy as set of
    connected markets
  \item
    Viewed operation of markets from the standpoint of individual agents
    (methodological individualism) who the best available means to
    achieve their goals

    \begin{itemize}
    \tightlist
    \item
      Competitive selection under constraint of scarcity
    \end{itemize}
  \item
    Results of individual decisions are aggregated by the market and
    manifest in the system of relative prices

    \begin{itemize}
    \tightlist
    \item
      Prices reshaped at the margin by the last decision →
      ``marginalism''
    \end{itemize}
  \item
    This approach to the economy had 2 fundamental motivations:

    \begin{itemize}
    \tightlist
    \item
      Cut through confusion about the relation between underlying value
      and price that had beset pre-marginalist, ``classical'' economics
    \item
      Create form of economic science relatively invulnerable to
      empirical controversy and ideological conflict
    \end{itemize}
  \item
    At time of marginalist revolution, there were many alternative
    directions for economics:

    \begin{itemize}
    \tightlist
    \item
      Edgeworth's program: economics as psychological and behavioral
      science
    \item
      Alfred Marshall: develop economics by analogy to natural history
      (weather or tides) -- context-bound, loose causal sequences
    \item
      These have resurfaced, but never broken hegemony of approach that
      resulted from marginalist turn
    \end{itemize}
  \end{itemize}
\item
  What are the limitations of this economics? 4 main ones:

  \begin{itemize}
  \tightlist
  \item
    Relative dissociation of formal analysis from causal and empirical
    investigation

    \begin{itemize}
    \tightlist
    \item
      Less a causal science than a form of logic
    \item
      In its purest form, it's an analytic apparatus, innocent of
      factual stipulations and normative commitments but also of causal
      theories other than residual causal theories that could be
      inferred from simple scheme of means-ends rationality under
      constraint of scarcity
    \item
      Marginalist theoreticians exploited affinity of that scheme to
      deductive reasoning
    \item
      This affinity helps explain why this economics reveres
      mathematics, but almost all math deployed in economics is
      relatively primitive

      \begin{itemize}
      \tightlist
      \item
        Not because economists are incapable, but because higher
        mathematics is useless to their method
      \item
        Only a relatively simple mathematics is needed to carry the work
        of this quasi-logical science in which means-ends scheme is the
        basis of analogy to syllogistic reasoning
      \end{itemize}
    \item
      Characteristic procedure of economists is development and
      substitution of models to explain a set of economic phenomena

      \begin{itemize}
      \tightlist
      \item
        If model doesn't work, create another one
      \item
        At no point does substitution of models jeopardize underlying
        theory, because theory at its heart is not a causal theory
        making empirical statements -- it's a logical scheme supporting
        deductive reasoning
      \item
        Different from, e.g., Standard Model of particle physics
      \item
        If Standard Model faced accumulating level of contrary empirical
        observation, there would be some attempt to ``save'' it by
        rearrangements among its propositions, or by context-bound
        qualifications, but would be swept aside
      \item
        Can never happen in marginalist economics because there's
        nothing in the elementary schema that anything that in the world
        could fundamentally impugn
      \end{itemize}
    \item
      This form of analytic invulnerability is an onus preventing
      dialectic of theory and discovery, of counterintuitive insight
      into workings of the world, condemning discipline to eternal
      infancy

      \begin{itemize}
      \tightlist
      \item
        But hasn't economics become increasingly empirical? ``Copious
        flesh hangs awkwardly from a frail skeleton''
      \item
        Where does causal element come from? Needs to come from
        means-ends rationality or be imported from a different
        discipline -- behavioral or neuroeconomics (developing what was
        earlier Edgeworth's program)
      \item
        Or developed on the spot -- context-bound economics
      \end{itemize}
    \end{itemize}
  \item
    Deficit of institutional imagination

    \begin{itemize}
    \tightlist
    \item
      Coexistence between 3 types of economics:

      \begin{itemize}
      \tightlist
      \item
        Pure economics: empty of institutional assumptions/commitments,
        exemplified by general equilibrium analysis
      \item
        Fundamentalist economics: Hayek, associating abstract idea of
        market economy with contingent set of institutional and legal
        arrangements (contract, property)
      \item
        Equivocating economics: macroeconomics, professes to study
        lawlike relations among large-scale aggregates against shadowy
        institutional background. Change in details of background
        matter, but because background is static, these details can be
        disregarded
      \end{itemize}
    \item
      To the extent this economics is rigorous, it is empty of
      institutional insight; to the extent it has institutional
      implications, it has them by virtue of equivocation and confusion
    \end{itemize}
  \item
    Lacking proper account of production

    \begin{itemize}
    \tightlist
    \item
      Before: theory of exchange and of production had same importance
    \item
      Now: economics textbook chapter on production has almost nothing
      about what we would ordinarily call production

      \begin{itemize}
      \tightlist
      \item
        Almost everything is about factors of production and
        substitution, different forms of market organization
      \end{itemize}
    \item
      Reduction of production to shadowy extension of exchange made
      possible by a contingent feature of economies: in them, labor can
      be bought and sold (what Marx considered the most important
      feature of capitalism)

      \begin{itemize}
      \tightlist
      \item
        So production can be seen under lens of relative prices
      \end{itemize}
    \end{itemize}
  \item
    Theory of competitive selection bereft of any account of how
    material from which the method selects is formed

    \begin{itemize}
    \tightlist
    \item
      Like if we had only Darwinian natural selection and not the part
      about genetic recombination
    \item
      Fecundity of process depends on richness of material on which it
      operates
    \item
      No account of economic life is acceptable if it's missing that
      half
    \end{itemize}
  \end{itemize}
\item
  Address 2 largely halfhearted and failed attempts to break out of
  trajectory that marginalist revolution established

  \begin{itemize}
  \tightlist
  \item
    Development economics

    \begin{itemize}
    \tightlist
    \item
      Albert Herschmann: potential to represent alternative direction
      for economic theory, but pursued strategy of peaceful coexistence
      with dominant paradigm, eventually swept aside, reduced to series
      of context-dependent studies

      \begin{itemize}
      \tightlist
      \item
        Squandered opportunity to be point of departure for
        consequential intellectual rebellion
      \end{itemize}
    \end{itemize}
  \item
    Keynesianism

    \begin{itemize}
    \tightlist
    \item
      Most successful economic apostate of 20th century, but a contained
      apostasy: insufficient to serve as basis for program of critique
    \item
      Almost no institutional content -- key variables all
      psychological: preference for liquidity, propensity to consume,
      long-term expectations

      \begin{itemize}
      \tightlist
      \item
        Tendency in English political economy of psychologism, prior to
        marginalism
      \item
        Institutional observation limited to particular domains like
        stock market
      \end{itemize}
    \item
      Focus on demand side, rather than supply side

      \begin{itemize}
      \tightlist
      \item
        Can't have alternative economics without comprehensive theory of
        breakthroughs on supply side and demand side (and relation
        between the two)
      \end{itemize}
    \item
      Theory of economic slumps was not a ``general theory,'' but one of
      a particular kind -- characterized by hoarding, rigidity of wages,
      inadequate demand

      \begin{itemize}
      \tightlist
      \item
        Inadequacy manifest in difficulty we have applying it to
        recessions of present day: leverage, expansion of credit masking
        absence of property-owning democracy, concentration of wealth
        resulting in asset bubbles producing vulnerability that begins
        in finance, but compromises real economy
      \end{itemize}
    \item
      Caught in between theory of equilibrium at low level of employment
      (which American followers adopted) and permanent disequilibrium
    \item
      Lesson in dangers of partial intellectual rebellion. 4 stages:

      \begin{itemize}
      \tightlist
      \item
        Keynes focused on response to slump that he regarded as
        politically palatable: only focus on demand side
      \item
        Year later, his follower Hicks developed IS-LM schema --
        formulaic reduction of his doctrine
      \item
        American followers (led by Samuelson) reinterpreted, downsized
        theory into theory of fiscal and monetary policy for
        countercyclical management of economy
      \item
        Labeled ``macroeconomics'' and placed alongside inherited body
        of theory (redubbed ``microeconomics'')
      \item
        Under headings like ``microfoundations of macroeconomics,''
        economists observed everything new in this theory was
        unnecessary and could be inferred from microeconomic/marginalist
        theory
      \end{itemize}
    \end{itemize}
  \item
    After all of this, we have no way of thinking structurally about the
    economy

    \begin{itemize}
    \tightlist
    \item
      Leftists embraced vulgar Keynesianism when they abandoned Marx
    \end{itemize}
  \end{itemize}
\item
  No alternative to a confrontation

  \begin{itemize}
  \tightlist
  \item
    Economics is useful, even indispensable as science of tradeoffs and
    constraints, friend of intellectual clarity and rigor
  \item
    But it can't bring us to the threshold of insight into structural
    alternatives
  \item
    How to proceed? Most ambitious: a comprehensive theory. But this is
    the limiting case
  \end{itemize}
\end{itemize}

\textbf{Rodrik}

\begin{itemize}
\tightlist
\item
  This picture of what an alternative economics might look like is
  setting standards we can't meet -- possible advantages are outweighed
  by disadvantages

  \begin{itemize}
  \tightlist
  \item
    Mistakes RMU cited: those economists were looking for universal laws
    or a theory of history that also wants economics to do
    \textit{too much}
  \end{itemize}
\item
  Economists have carved out useful function by performing a certain
  trick: study of a method, rather than study of the economy

  \begin{itemize}
  \tightlist
  \item
    ``Thinking like an economist'' has compelling power
  \item
    Sometimes taken too far -- economic imperialism
  \end{itemize}
\item
  Main disagreement: what RMU is looking for is something universal to
  think about structural alternatives

  \begin{itemize}
  \tightlist
  \item
    But not job of economics to provide that
  \item
    Smith and Marx had a particular view of what economic progress might
    look like

    \begin{itemize}
    \tightlist
    \item
      Universalist view ended up backfiring
    \end{itemize}
  \end{itemize}
\item
  Approach is immune to challenge? True that if particular model is
  challenged, it's not the way of thinking that's challenged, it's the
  model
\item
  Economics is different than natural sciences because in principle
  there is an infinite variety of social reality to describe -- no
  universal constants

  \begin{itemize}
  \tightlist
  \item
    Cause and effect relationship is always contingent in a potentially
    infinite number of ways
  \item
    How can something so indeterminate be useful?

    \begin{itemize}
    \tightlist
    \item
      Ex: How would minimum wages work? Depends on competitive
      equilibrium vs.~monopsony. It depends, but we know what it depends
      on, and we can empirically check which model is relevant
    \end{itemize}
  \end{itemize}
\item
  Economics has moved from programmatic vision (Smith, Marx) to a
  discipline -- a deductive, analytical tool

  \begin{itemize}
  \tightlist
  \item
    Can distinguish analytical thinking from normative/ideological
    baggage
  \item
    Normative substance needs to come from elsewhere

    \begin{itemize}
    \tightlist
    \item
      Chicago school libertarians: combined tools with normative
      judgements of inequality and practical view of government as
      corrupt
    \item
      Krugman-type liberals: same tools, combine with different
      normative judgements on inequality and more hopeful view of
      government
    \end{itemize}
  \end{itemize}
\item
  Failings of economics:

  \begin{itemize}
  \tightlist
  \item
    Mistake model for reality
  \item
    Categorical preference for certain axioms (e.g.~rationality)
  \item
    Preference for problems that are amenable to these tools
  \item
    Implicit political-economy theorizing in policy discussions

    \begin{itemize}
    \tightlist
    \item
      Especially in focus on efficiency as only way to evaluate policies
    \end{itemize}
  \end{itemize}
\item
  What RMU calls a shortcoming is an advantage -- don't want economists
  to lay claim to universal framework for thinking about structural
  alternatives
\end{itemize}

\textbf{Unger}

\begin{itemize}
\tightlist
\item
  Task of theory is to provide insight into transformation

  \begin{itemize}
  \tightlist
  \item
    To understand something is to understand what it can become
  \end{itemize}
\item
  If we use models to represent established economic phenomena, how do
  we understand how they change?

  \begin{itemize}
  \tightlist
  \item
    In post-marginalist tradition, theory is doing something else:
    exempting itself from insight into transformation and providing
    supposedly neutral analytical framework
  \item
    Role that would be theory's is played by ``art''

    \begin{itemize}
    \tightlist
    \item
      ``Intuition for model selection''
    \item
      This shouldn't exempt us from having a view on transformation
    \end{itemize}
  \end{itemize}
\item
  Criticize invocation of intellectual modesty: ``someone else will deal
  with structural matters''

  \begin{itemize}
  \tightlist
  \item
    In current day, central space of social theory and philosophy is
    vacant
  \item
    Real debate happens in economics and law
  \item
    Both barred themselves against responsibility of transformative
    insight

    \begin{itemize}
    \tightlist
    \item
      Economics as a result of history of marginalism
    \item
      Law because it's representing system of idealized policies and
      principles
    \end{itemize}
  \item
    Both disciplines are misdirected -- but we say someone else will do
    this work of imagining alternatives

    \begin{itemize}
    \tightlist
    \item
      ``Someone else'' has to be us
    \end{itemize}
  \end{itemize}
\end{itemize}

Q: A lot of legal thinking based on categorical morality. Economic
thinking based on consequentialism: individual utility maximization, or
social welfare. Is there a tension? How do we reform economic models?

\textbf{Unger}

\begin{itemize}
\tightlist
\item
  Diminishing marginal returns is pretty close to a law in economic
  thinking

  \begin{itemize}
  \tightlist
  \item
    If we could loosen or reverse this, we would have potential for
    exponential growth
  \item
    If we think of innovation as functionally equivalent to input and
    it's continuous rather than episodic, then we have potential to
    escape diminishing marginal returns
  \end{itemize}
\item
  No rich background of conversations about transformation of structure
\item
  Phillips Curve stops working → something changed in institutional
  background → no way of thinking about relation between economic
  phenomena and institutional background. Another way we might begin
\item
  We don't have to treat this as prerogative of genius
\item
  Can't just enrich and contextualize economics. Look to natural science
  -- rich set of causal conjectures and a physical or social conception
  of how something transforms

  \begin{itemize}
  \tightlist
  \item
    Math may or may not be useful
  \item
    But not toy mathematics that is simply the adornment of deductive
    reasoning

    \begin{itemize}
    \tightlist
    \item
      Sterile as methodological basis for thinking about transformation
    \end{itemize}
  \end{itemize}
\end{itemize}

\hypertarget{conclusions-rethinking-political-economy}{%
\section{Conclusions: Rethinking political
economy}\label{conclusions-rethinking-political-economy}}

\textbf{Rodrik}

\begin{itemize}
\tightlist
\item
  Antecedent model: New Deal/welfare state, settlement between working
  class (accumulated through Industrial Revolution) and
  business/financial elites

  \begin{itemize}
  \tightlist
  \item
    Significant social insurance
  \item
    Job security through institutionalized labor markets (trade unions,
    collective bargaining)
  \item
    Regulated finance
  \item
    Mass production
  \item
    Keynesian macroeconomic management -- activist fiscal, monetary
    policy targeting stabilization of aggregate demand
  \item
    ``Thin'' globalization: Bretton Woods and IMF = capital controls,
    GATT system
  \end{itemize}
\item
  Collapse of this model:

  \begin{itemize}
  \tightlist
  \item
    Technological change with unequal rewards
  \item
    Hyperglobalization
  \item
    Financial globalization: free flow of short-term capital
  \item
    Deinstitutionalization of labor markets
  \item
    Deindustrialization (premature in low- and middle-income countries)
  \item
    Neoliberal narrative on macroeconomic management: downplayed active
    management, emphasized markets, austerity
  \end{itemize}
\item
  Empirically: decline of middle-class share of income; employment
  prospects and wage of middle class has suffered most (worldwide)
\item
  Three interconnected challenges:

  \begin{itemize}
  \tightlist
  \item
    Inclusion: how do we ensure those who have been excluded from
    advanced technologies are reincorporated?
  \item
    Growth: address slowdown of productivity growth in advanced
    countries, premature deindustrialization in developing countries
  \item
    (Political: authoritarianism, personalistic strong-man politics)
  \end{itemize}
\item
  Common root: scarcity of productive, stable middle-class jobs
\item
  Remedies?

  \begin{itemize}
  \tightlist
  \item
    New social contract based on new ``industrial'' policy

    \begin{itemize}
    \tightlist
    \item
      Increase supply of good jobs
    \item
      Redirect innovation in labor-friendly direction
    \end{itemize}
  \item
    State strong enough to achieve inclusive growth, but not strong
    enough to smother pluralism
  \item
    Balanced globalization. Encourage pluralism, but focus on:

    \begin{itemize}
    \tightlist
    \item
      Global public goods (pandemics, climate change)
    \item
      Beggar-thy-neighbor policies (global tax havens)
    \item
      Areas with large economic gains (labor mobility)
    \end{itemize}
  \end{itemize}
\end{itemize}

\textbf{Unger}

\begin{itemize}
\tightlist
\item
  Transition in most advanced practice of production: traditional mass
  production to knowledge economy

  \begin{itemize}
  \tightlist
  \item
    Smith and Marx understood that best way to understand alternative
    futures is to look to most advanced practice
  \item
    Transition has produced a disappointment and a danger:

    \begin{itemize}
    \tightlist
    \item
      Appears as series of exclusive fringes
    \item
      Leads to economic stagnation, aggravation of inequality
    \end{itemize}
  \item
    How do we deepen and disseminate the most advanced practice?
  \end{itemize}
\item
  Broader focus: situation and sequel to last major moment of
  institutional and ideological refoundation -- the social
  democratic/social liberal settlement developed after WWII

  \begin{itemize}
  \tightlist
  \item
    In US: New Deal welfare state
  \item
    Regulation, attenuate inequality retrospectively (tax-and-transfer),
    countercyclical management
  \item
    Most important problems can't be addressed within the limits of this
    compromise
  \end{itemize}
\item
  3 problems in political economy:

  \begin{itemize}
  \tightlist
  \item
    Relation of backward to advanced practices of production
  \item
    Relation of labor to capital
  \item
    Relation of finance to production
  \end{itemize}
\item
  Social cohesion: money transfers organized by state against background
  of ethnic homogeneity

  \begin{itemize}
  \tightlist
  \item
    As homogeneity fades, inadequacy of money as social cement is
    exposed
  \end{itemize}
\item
  Under arrangement of weak democracies, change continues to depend on
  crisis

  \begin{itemize}
  \tightlist
  \item
    No trauma, no transformation
  \item
    Need form of democratic politics that diminishes dependence of
    change on crisis
  \end{itemize}
\item
  Methods: we have a problem in contemporary thought with the discussion
  of structural discontinuity, structural alternatives

  \begin{itemize}
  \tightlist
  \item
    Classical thought had structural vision which was corrupted by
    various forms of determinism (Marxism)
  \item
    Contemporary social sciences: deficient in structural imagination
  \item
    Embarrassment in programmatic arguments: feasible but trivial
    vs.~interesting but utopian → everything proposed likely to be
    dismissed as trivial or utopian

    \begin{itemize}
    \tightlist
    \item
      Fall back on bastardized criterion of political realism: proximity
      to the existent
    \item
      False dilemma threatens to inhibit the programmatic ambition
    \end{itemize}
  \end{itemize}
\item
  Hope: past great thinkers had generous vision of unrealized human
  opportunity motivating search for transformation

  \begin{itemize}
  \tightlist
  \item
    This whole course is a bet on the marriage of insight to hope
  \end{itemize}
\item
  The only democracies we have are weak and low-energy

  \begin{itemize}
  \tightlist
  \item
    Based on low level of people's engagement in political life
  \item
    Awaken only when there's trouble
  \end{itemize}
\item
  Not persistent tests of established structure
\item
  Alternatives: authoritarian vanguards that claim to speak in name of
  collective interest, but hold this collective interest ransom to their
  dogmas
\item
  Transform weak democracies into strong democracies: depends on
  institutional innovations in the form of democratic politics

  \begin{itemize}
  \tightlist
  \item
    Increase level, tempo of politics
  \end{itemize}
\item
  Basic schema of ideological controversy: state against the market
  (more of one or the other, or synthesis of market and state)

  \begin{itemize}
  \tightlist
  \item
    Different conception of ideological contest: alternative
    institutional form of the market, independent civil society
  \end{itemize}
\item
  World is restless under dictatorship of no alternatives
\end{itemize}
